\begin{aufgabe}
Man simuliere die Bewegung der Sterne in einem Kugelsternhaufen.
\end{aufgabe}

Wikipedia:
\begin{quote}
Ein Kugelsternhaufen (kurz auch Kugelhaufen) ist eine enge,
kugel\-f"ormige Ansammlung sehr vieler Sterne, die untereinander
gravitativ gebunden sind. Typische Gr"ossen sind einige 100000 Sterne.
Gegenseitige Bahnver"anderungen sind im dicht bev"olkerten Zentrum
häufig, was die sph"arische Gestalt zur Folge hat. Kugelsternhaufen
sind ihrerseits gravitativ an Galaxien gebunden, in deren Halo sie
sich weiträumig bewegen. Sie bestehen vorwiegend aus alten, roten
Sternen, die nur wenige schwere Elemente enthalten.
\end{quote}

Ziel dieser Aufgabe ist eine Simulation der Bewegung der Sterne in
einem Kugelsternhaufen durchzuf"uhren. Dabei k"onnten folgende Fragen
von Interesse sein:
\begin{enumerate}
\item Wie h"angt die Sterndichte nahe dem Zentrum des Kugelsternhaufens
von der Zeit ab?
\item Wie lange dauert es, bis der Kugelsternhaufen einen stabilen
Radius gefunden hat?
\item Wie h"aufig kommt es vor, dass Sterne aus dem Kugelsternhaufen
ausgestossen werden?
\item Wie "andert ein schwarzes Loch mit 20000 Sonnenmassen im Zentrum
eines Kugelsternhaufens die Antworten zu obigen Fragen?
\end{enumerate}
