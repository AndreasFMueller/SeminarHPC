\begin{aufgabe}
Beschleunigung der R"uckprojektion in einem CAT Scanner.
\end{aufgabe}

Eine CAT Scanner oder Computer-Tomograph ist in der Lage, aus
einzelnen linearen R"ontgenbildern in einer Ebene durch einen
K"orper ein Schnittbild durch den K"orper zu gewinnen.
CAT Scanner wurden in den 60er und 70er Jahren entwickelt
und haben die medizinische Diagnostik revolutioniert. 

Die mathematische Aufgabe hinter einem CAT Scanner kann wie folgt
beschrieben werden.
Wir m"ochten ein Schnittbild durch eine K"orper erstellen,
wir m"ochten also die Absorptionsdichte in Abh"angigkeit $f(x,y)$ von 
$x$ und $y$ ermitteln.
Zur Verf"ugung stehen uns R"ontgenbilder, die in verschiedenen Richtungen
aufgenommen wurden.
Der R"ontgenstrahl durch den Punkt $(r\cos\vartheta, r\sin\vartheta)$
mit der Normalen $(\cos\vartheta,\sin\vartheta)$ wird abh"angig
von $f$ abgeschw"acht, die 
gesamte Absorbtion entlang dieses Strahls ist gegeben durch das
Integral entlang dieses Strahls.
Den Strahl kann man in Parameterdarstellung beschreiben:
\[
\begin{pmatrix}
r\cos\vartheta\\
r\sin\vartheta
\end{pmatrix}
+s
\begin{pmatrix}
-\sin\vartheta\\
\cos\vartheta
\end{pmatrix}
\]
Die aufgenommenen R"ontgenbilder haben also Absorbtionen
\[
{\cal R}f(r,\vartheta)=\int_{-\infty}^{\infty}
f(r\cos\vartheta-s\sin\vartheta, r\sin\vartheta+s\cos\vartheta)\,ds.
\]
Die Funktion ${\cal R}f(r,\vartheta)$ heisst die Radon-Transformierte
von $f$, nach Johann Karl August Radon (1887--1956), der diese Transformation
bereits 1917 studiert hat.

Die Aufgabe eines Computer-Tomographen ist also, aus ${\cal R}f$ die
urspr"ungliche Funktion $f$ zu rekonstruieren.
Ein wesentlicher Schritt dazu ist die sogenannte R"uckprojektion.
Ziel dieser Aufgabe ist herauszufinden, wie gut sich die R"uckprojektion
mit Hilfe von Graphikkarten beschleunigen l"asst.

Um den Wert $f(x,y)$ zu finden, muss man sicher alle Werte von ${\cal R}f$
verwenden, die zu Geraden geh"oren, die durch den Punkt $(x,y)$ gehen.
Mittel man "uber alle Werte von ${\cal R}f(r,\vartheta)$, die zu 
Geraden durch $(x,y)$ geh"oren, dann werden sich die Werte von anderen
Pixeln ausmitteln zu einem mittleren Wert. Man kann sogar eine
Formel daf"ur angeben:
\[
{\cal B}g(x,y)
=
\frac1\pi\int_0^\pi g(x\cos\vartheta+y\sin\vartheta,\vartheta)\,d\vartheta.
\]
Diese R"uckprojektion ist also eine erste, noch etwas verschleierte
N"aherung f"ur $f(x,y)$, durch weitere Verarbeitungsschritte, die
nicht Gegenstand dieser Aufgabe sind, kann man das Bild auch noch
scharf hinbekommen.

Literatur: Timothy G.~Feeman, {\it The Mathematics of Medical Imaging:
A Beginners Guide}, Springer Undergraduate Texts in Mathematics and
Techology, 2010.

