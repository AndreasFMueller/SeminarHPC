\begin{aufgabe}
Str"omungssimulation mit OpenFOAM.
\end{aufgabe}

OpenFOAM ist ein Open Source Softwarepaket zur Str"omungssimulation.
Ziel dieser Aufgabe ist darzulegen, wie ein erfolgreiches Produkt
wie OpenFOAM die verschiedenen Techniken einsetzt, die im 
Vorlesungsteil besprochen worden sind.

Als Basis f"ur diese Diskussion soll ein einfaches Str"omungsproblem
gel"ost werden.
Es soll die Str"omung um einen Fl"ugel mit einem sehr flachen Rhombus
als Querschnitt berechnet werden. 
Folgende Fragen k"onnten dabei von Interesse sein:
\begin{enumerate}
\item Wie h"angen Auftrieb und Widerstand vom Anstellwinkel ab?
\item Bei welchem Anstellwinkel beginnt sich die Str"omung abzul"osen?
\end{enumerate}

