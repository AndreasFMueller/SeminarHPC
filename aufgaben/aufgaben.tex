%
% aufgaben.tex -- Kurzbeschreibungen der Aufgabenstellungen fuer die
%                 Seminar-Arbeiten
%
% (c) 2013 Prof Dr Andreas Mueller, Hochschule Rapperswil
% $Id$
%
\documentclass[a4paper,12pt]{article}
\usepackage{german}
\usepackage{amsmath}
\usepackage{amssymb}
\usepackage{amsfonts}
\usepackage{amsthm}
\usepackage{graphicx}
\usepackage{fancyhdr}
\usepackage{textcomp}
\usepackage[all]{xy}
\usepackage{txfonts}
\usepackage{alltt}
\usepackage{verbatim}
\usepackage{paralist}
\usepackage{makeidx}
\usepackage{array}
\begin{document}
\title{Aufgabenstellungen f"ur das Seminar\\``High Performance Computing''}
\author{Andreas M"uller\footnote{
University of Applied Sciences, Oberseestrasse 10, CH-8640 Rapperswil,
Switzerland, Email: {\tt andreas.mueller@hsr.ch}}}
\date{}
\maketitle
\section{Ziele des Seminars}
\begin{itemize}
\item
\item
\item
\item
\end{itemize}

\section{Auftrag}
Jeder Seminarteilnehmer bearbeitet ein Thema aus dem Bereich des
High Performance Computing,
stellt seine Resultate in Form eines
kurzen Papers (wenige Seiten) zusammen und stellt sie ausserdem
in Form einer Pr"asentation im Klassenrahmen vor.

Im Idealfall l"asst sich Ihr Paper fast unver"andert als Abschnitt
in das Skript aufnehmen, so dass den Teilnehmern am Ende des Seminars
ein kleiner ``Leitfaden zum High Performance Computing'' zur Verf"ugung steht,
der als Einstieg in die Fachliteratur dienen kann.
Das Skript im Source Code wird als Creative Commons Projekt auf
Github gefuhrt, unter dem URL
{\tt https://github.com/AndreasFMueller/SeminarHPC.git}.

Die Pr"asentation sollte nicht nur einfach den Stoff vorf"uhren,
sondern sollte gen"ugend Zahlenbeispiele, Aufgaben oder von Hand
durchf"uhrbare Schritte einhalten, dass die Zuh"orer sich durch
die Probleml"osung durcharbeiten und so ihr Verst"andnis des
Stoffes vertiefen k"onnen. Die Pr"asentation soll sich auch darum
bem"uhen, den Zusammenhang mit den anderen im Seminar behandelten
Themen herzustellen, also zum Beispiel darlegen, warum eine bestimmte
Methode in gewissen F"allen einer anderen, von jemand anderem behandelten
Methode vorzuziehen ist.

Die Darstellung soll vor allem verst"andlich und anschaulich
sein, aber nat"urlich auch so exakt wie m"oglich.
Sie muss alllerdings nicht die strengen Anforderungen an einen
vollst"andigen Beweis erf"ullen. Es ist besser, sich auf die
wesentlichsten F"alle zu beschr"anken, oder die Methode mit
einem niedrigdimensionalen Beispiel zu illustrieren, welche
die Methode verst"andlich macht, als den Zuh"orer mit stundenlangen
Falldiskussionen zu langweilen.

\section{Aufgaben}
Die nachstehend skizzierten Aufgaben werden in dieser Reihenfolge im
Seminar behandelt. Die genauen Termine werden sp"ater bekannt
gegeben. Als Faustregel dient, dass Aufgabe Nummer $n$ in der Woche
$n + 2$ behandelt wird.

\newtheorem{aufgabe}{Aufgabe}

\section{Voraussetzungen}

\section{Bewertung}
\subsection{Note}
Die Vortr"age und Papers werden benotet und geben die Modulnote.
Die Note setzt sich zusammen aus den Resultaten eines Fragebogens,
den die Zuh"orer ausf"ullen, und einer Benotung durch den Dozenten.

\subsection{Kurztests}
Ausserdem werden drei Kurztests durchgef"uhrt, die nur bestanden
werden k"onnen oder nicht. 
{\em Nichtlineare Optimierung} und {\em Variationsrechnung} je
ein Kurztest bestehend aus einer Standardaufgabe aus den jeweiligen
Themengebieten verlangt. In diesem Kurztest d"urfen beliebige Hilfsmittel
verwendet werden, auch Computer und das Internet, und es gibt keine
Zeitlimite (ausser der Schliessung des Ge"audes).

Mit dem Kurztest
soll bewiesen werden, dass jeder Teilnehmer mit den grundlegenden
Techniken des HPC vertraut ist.
Die Kurztests geben keine Note, m"ussen aber alle erf"ullt sein,
damit das Modul als bestanden gilt.

\end{document}
 
