\begin{aufgabe}
Ising-Modell
\end{aufgabe}

Das Ising-Modell beschreibt einen Ferromagneten, bestehend aus 
Spins $S_i$, die in einem Gitter angeordnet sind. Die Energie
in ist gegeben durch
\[
H(S)=-J\sum_{\text{$i,j$ benachbart}}S_iS_j + B\sum_i S_i
\]
Die Statistische Mechanik liefert einen Formalismus, wie man thermodynamische
Eigenschaften dieses Systems berechnen kann. Dazu muss man die Zustandssumme
\[
Z_\beta = \sum e^{-\beta S}
\]
berechnen, wobei $\beta=1/k_BT$ die inverse Temperatur ist.
Die Summe erstreckt sich "uber alle m"oglichen Zust"ande, und genau das
ist das Problem: ein $100\times 100 \times 100$ Gitter, also immer noch
ein sehr kleiner Kristall, hat $2^{100^3}\simeq 10^{301029}$ Zust"ande,
diese Summe kann also praktisch nicht berechnet werden.

Eine Simulation ist n"otig.
Es existiert eine Reihe von Algorithmen, mit denen ein Gleichgewichtszustand
gefunden werden kann.
W"ahlen Sie einen Algorithmus und implementieren Sie eine parallelisierte
Version desselben.

\begin{itemize}
\item
\url{http://en.wikipedia.org/wiki/Ising_model}
\item
\url{http://www.comphys.ethz.ch/hans/p/047.pdf}
\item
\url{http://physics.unifr.ch/admin/dbproxy.php?table=fuman_filepool&column=content&id=1267}
\end{itemize}

