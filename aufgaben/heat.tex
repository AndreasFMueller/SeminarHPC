\begin{aufgabe}
Man l"ose die W"armeleitungsgleichung
\[
\frac{\partial u}{\partial t}
=
\kappa
\biggl(
\frac{\partial^2u}{\partial x^2}
+
\frac{\partial^2u}{\partial y^2}
\biggr)
=
\kappa\Delta u,
\]
numerisch auf dem Gebiet $\Omega=\{(x,y)\,|\,0\le x,y\le 1\}$ und
$t\ge 0$ mit Anfangsbedingung $u(0, x, y)=g(x,y)$ und Randbedingung
$\frac{\partial u}{\partial n}(t,x,y)=0$.
\end{aufgabe}

Die W"armeleitungsgleichung hat nur erste Ableitungen in der Zeit. 
Dies hat zur Folge, dass die diskretisierte Gleichung mit einem
sogenannten Time-Marching-Algorithm f"ur fortschreitende Zeit
gel"ost werden kann.
Ziel dieser Aufgabe ist zu zeigen, wie diese
diskretisierten Gleichungssysteme aussehen, und wie sie f"ur eine sehr
grosse Zahl von Knoten im Gebiet $\Omega$ gel"ost werden k"onnen.



