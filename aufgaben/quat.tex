\begin{aufgabe}
Das quaternionische Apfelm"anchen.
\end{aufgabe}

In der Menge der Quaternionen\footnote{Quaternionen erweitern die
komplexen Zahlen zu einer vierdimensionalen Algebra. Sie sind unter anderem
n"utzlich zur Beschreibung von Drehungen des Raumes.
Die Quaternionenalgebra ist nicht kommutativ.}
$\mathbb H$ kann man die Iteration
der quadratischen Funktion
\[
f_c\colon\mathbb H\to\mathbb H\colon
z\mapsto f_c(z)=z^2+c
\]
f"ur verschiedene Werte von $c$ untersuchen. Wie bei der komplexen
quadratischen Iteration werden Parameterwerte $c$ danach klassiert,
ob die Iteration $z_i=f_c(z_{i-1})$ divergiert oder beschr"ankt bleibt.

Beschr"ankt man $c$ auf komplexe Zahlen, bleiben die Iterierten ausgehend
vom Punkt $0$ alle komplex, die Menge der Parameterwerte $c$, f"ur die
die Iteration beschr"ankt bleibt ist die Mandelbrot-Menge (das Apfelm"annchen). 
Wie sieht die Menge
\[
M=\{c\in\mathbb H\,|\,f_c^i(0)\text{ beschr"ankt}\}
\]
aus?

