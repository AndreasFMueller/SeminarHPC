Sei $A_a$ die Matrix
\[
A_a=\begin{pmatrix}a&1\\1&a\end{pmatrix}.
\]
F"ur welche Werte von $a$ ist das Jacobi-Verfahren f"ur diese Matrix
konvergent?

\begin{loesung}
Das Jacobi-Verfahren entspricht der Zerlegung
\[
M_a=\begin{pmatrix}a&0\\0&a\end{pmatrix},\qquad
N=\begin{pmatrix}0&1\\1&0\end{pmatrix}.
\]
Die Matrix
\[
C=M^{-1}N=\frac1a\begin{pmatrix}0&1\\1&0\end{pmatrix}
\]
hat das charakteristische Polynom
\[
\chi_C(\lambda)=\lambda^2-\frac1{a^2}
\]
mit den Nullstellen $\lambda_\pm=\pm\frac1a$. Insbesondere gilt
\[
\varrho(M^{-1}N)=\frac1{|a|}\begin{cases}
>1\qquad\qquad &|a| < 1,\\
=1\qquad &|a|=1,\\
<1\qquad &|a| > 1.
\end{cases}
\]
Das Verfahren ist also genau f"ur $|a|>1$ konvergent.
\end{loesung}

