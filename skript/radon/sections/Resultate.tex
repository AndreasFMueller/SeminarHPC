\section{Resultate}

Im Vergleich der sequentiellen- und der paralellen Implementation
zeigt sich ein deutlicher Unterschied. F"ur die Tests verwendeten wir
abwechslungsweise eine CPU und eine GPU und liessen jeweils ein $512\times
512$ Pixel Bild rechnen. 

CPU: Intel Core i7 - 3610QM (4C/8T) 

GPU: nVidia Tessla (Grafikkarte) 

Die Tests brachten erstaunliche Ergebnisse. Die sequentielle Variante
lastete einen Kern der CPU vollst"andig aus, die parallele Variante
lastete alle 4 Kerne und alle 8 Treads der CPU aus, war dabei jedoch
beinahe vier mal schneller. Die GPU war sogar beinahe 50 mal schneller
als die sequentielle Implementation.

Rechenzeit f"ur ein $512\times 512$ Pixel Bild:
\begin{center}
\begin{tabular}[b]{|c|c|c|c|}
	\hline
	Hardware & sequentiell & paralell \\
	\hline \hline
	CPU & 0.91 s / 1.1 fps & 0.25 s / 4 fps \\
	\hline
	GPU & NA & 0.02 s / 50 fps \\
	\hline
\end{tabular}
\end{center}
	
Nach Analyse der Rechenoperationen und der Rechenzeiten, l"asst
sich schliessen, dass die Rechenzeit vor allem von den vorhandenen
Floatigpointunits abh"angt. Dies erkennt man sehr sch"on anhand der
CPU Zeiten, denn der i7 besitzt vier FPU's. Diese Rechnung ist jedoch
nicht allgemein g"ultig da immer auch ein gewisser Overhead mit jeder
FPU in die Rechenzeit mit einfliesst. Aber dennoch gilt allgemein,
je mehr FPU's desto schneller.



  
