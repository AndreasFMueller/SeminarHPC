\section{Radontransformation und R"uckprojektion}

Durch eine Funktion $f(x,y)$, welche die Absorbtion der R"ontgenstrahlen
in einem bestimmten Punkt im K"orper darstellt, erhalten wir die gesuchte
Dichte des K"orpers in diesem Punkt, da die Absorbtion direckt von der
Dichte des K"orpers zusammenh"angt.
Es ist jedoch nur m"oglich die Absorption entlang eines Strahles durch
den K"orper zu messen. Die Absorbtion entlang des geradlinigen Strahles
entspricht der Radontransformation $Rf(r,\vartheta)$ bei einem bestimmten
Winkel.
Der CAT-Scanner schiesst R"ontgenstrahlen aus allen Richtungen durch
den K"orper und rechnet dann grundlegend die Linien-Integrale "uber
jede einzelne Gerade aus. Die anfallenden Daten entsprechen der
Radontransformation $g(r,\vartheta) = Rf(r,\vartheta)$. 

Parameterdarstellung eines Strahls:
\begin{equation}
	r
	\begin{pmatrix}
		\cos{\vartheta} \\ \sin{\vartheta}
	\end{pmatrix}
	+ s
	\begin{pmatrix}
		-\sin{\vartheta} \\ \cos{\vartheta}
	\end{pmatrix}
\end{equation}

Radontransformation:
\begin{equation}
	Rf(r,\vartheta) = \int\limits_{-\infty}^{\infty} f(r\cos{\vartheta}-s\sin{\vartheta},r\sin{\vartheta}+s\cos{\vartheta})ds.
\end{equation}

Das Ziel der Arbeit ist es nun, aus der Radontransformation
$Rf(r,\vartheta)$ die Absortion $f(x,y)$ zur"uckzugewinnen.
Eine erste Approximation f"ur $f(x,y)$ kann man ziemlich gut durch die
Mittelwerte aller Geraden durch den Punkt $(x,y)$ bekommen.
Um dies zu erreichen m"ussen alle Werte von $Rf$ verwenden werden,
welche zur Geraden geh"oren, die durch den Punkt $(x,y)$ gehen. Wenn
man "uber alle Werte von $Rf(r,\vartheta)$ welche zur Geraden geh"oren
mittelt, dann werden sich die Werte von anderen Pixeln ausmitteln und zu
einem mittleren Wert f"uhren. Daf"ur kann man sogar eine Formel angeben,
die sogenannte R"uckprojektion.

R"uckprojektion:
\begin{equation}
	Bg(x,y) = \cfrac{1}{\pi}\int\limits_0^\pi g(x\cos{\vartheta}+y\sin{\vartheta},\vartheta) d\vartheta
\end{equation}

Durch diese Formel erhalten wir eine erste N"ahrung der
Radonr"ucktransformation. Wir k"onnen daraus das Orginalbild
zur"uckgewinnen, es ist lediglich verschleiert und weisst um helle
Stellen einen ebenfalls hellen Hof auf, auf welchen wir sp"ater noch
etwas genauer eingehen werden. Die erzeugten Bilder k"onnen nach jeder
Umdrehung zur"uckgerechnet und dargestellt werden.

