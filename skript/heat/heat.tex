\chapter{L"osung der W"armeleitungsgleichung}
\rhead{W"armeleitungsgleichung}

\chapterauthor{Andreas M"uller}

Die W"armeleitungsgleichung ist die prototypische parabolische
partielle Differentialgleichung.
In diesem Kapitel werden die Eigenheiten ihrer numerischen L"osung
diskutiert.

\section{Problemstellung}
\lhead{Problemstellung}
Die W"armeleitungsgleichung f"ur einen Stab
auf dem Interval $[0,1]$ ist die partielle Differentialgleichung
\begin{equation}
\frac{\partial u}{\partial t}=\frac{\partial^2 u}{\partial x^2}, \qquad
0<x<1,\;t>0.
\label{heat:pdgl}
\end{equation}
Der Stab soll an den Enden mit W"armereservoirs verbunden sein, die
die Temperatur an den Enden festlegen. Sie kann mit der Zeit varieren.
Dies wird ausgedr"uckt durch
Randbedingungen
\begin{align}
u(t,0)&=g_0(t),
&
u(t,0)&=g_1(t),&t&>0.
\label{heat:rand}
\end{align}
Die L"osung wird festgegelegt durch die Temperaturverteilung zu Beginn,
also durch
\begin{align}
u(0,x)=f(x),\qquad 0<x<1.
\end{align}
Gesucht ist ein numerisches L"osungsverfahren, welches die
W"armeleitungsgleichung numerisch l"ost.

Eine interessante und unphysikalische Eigenschaft der W"armeleitungsgleichung
ist die Existenz von L"osungen mit unendlich schneller Wirkungsausbreitung.
Die Funktion
\[
u(t,x)=\frac1{\sqrt{2t}}e^{-\frac{x^2}{4t}}
\]
hat die folgenden Ableitungen:
\begin{align*}
\frac{\partial u}{\partial x}&=
-
\frac{x}{t}
\frac1{\sqrt{2t}}
e^{-\frac{x^2}{4t}}
&
\frac{\partial^2 u}{\partial x^2}&=
\biggl(\frac{x^2}{4t^2}-\frac1{2t}\biggr)
\frac1{\sqrt{2t}}
e^{-\frac{x^2}{4t}}
\\
\frac{\partial u}{\partial t}&=
\biggl(\frac{x^2}{4t^2}-\frac1{2t}\biggr)
\frac1{\sqrt{2t}}
e^{-\frac{x^2}{4t}}
\end{align*}
Die Funktion $u$ ist also eine L"osung der W"armeleitungsgleichung
(\ref{heat:pdgl}).
F"ur $t\to 0$ strebt $u(t,x)$ gegen eine Dirac $\delta$-Funktion
im Punkt 0.
Dies bedeutet, dass W"arme, die zur Zeit $t=0$ nur im Punkte vorhanden
ist, unmittelbar danach f"ur alle $t$ in Form eines Wertes $u(t,x) > 0$
sp"urbar ist.
Die W"arme im Punkt $0$ hat sich also mit unendlich grosser Geschwindigkeit
ausgebreitet.

\section{Diskretisation\label{heat:section-diskretisation}}
Zur Diskretisation w"ahlen wir im Gebiet $[0,1]\times \mathbb R_+$ ein
Gitter. 
Da $x$- und $t$-Richtung offensichtlich nicht symmetrisch sind,
lassen wir offen, welche Gitterkonstanten wir in diesen Richtungen
verwenden wollen, und nennen Sie $h_x=1/n$ und $h_t$. 
Die Funktion $u(t,x)$ wird jetzt ersetzt durch die Werte von $u$ auf
den Gitterpunkten, wir schreiben
\[
u_{ij} = u(ih_t, jh_x),\quad i\ge 0, 0\le j \le n.
\]
Durch die Randbedingungen wird ein Teil dieser Variablen bereits
festgelegt:
\begin{align*}
u_{0j}&=f(jh_x)&1&\le x < n\\
u_{i0}&=g_0(ih_t)&i&>0\\
u_{i1}&=g_1(ih_t)&i&>0\\
\end{align*}
Die Diskretisation der Differentialgleichung verwendet die in
Abschnitt~\ref{subsection-motivation} Methode verwenden.
Die zweite partielle Ableitung von $u$ nach $x$ im Punkt $(ih_t, jh_x)$ 
kann durch
\begin{equation}
\frac1{h_x^2}(u_{i,j-1}-2u_{ij}+u_{i,j+1})
\label{heat:2ableitung}
\end{equation}
approximiert werden.
F"ur die erste partielle Ableitung nach $t$ ist man versucht,
den Differenzenquotienten
\begin{equation}
\frac1{h_t}(u_{i+1,j}-u_{ij})
\label{heat:1ableitung}
\end{equation}
zu verwenden.
Dieser Differenzenquotient ist aber eher eine Approximation f"ur
$\partial u/\partial t$ im Punkt $((i+\frac12)h_t, jh_x)$,
nicht $(ih_t,jh_x)$, man kann ihn also nicht direkt mit der
Approximation (\ref{heat:2ableitung}) f"ur die zweite Ableitung
in Beziehung setzen.
Man kann aber durch Mittelwertbildung aus (\ref{heat:2ableitung}) f"ur
die Punkte $(ih_t, jh_x)$ und $((i+1)h_t, jh_x)$ einen N"aherungswert
f"ur die zweite Ableitung finden, der sich mit der Approximation
f"ur die ersten Ableitung (\ref{heat:1ableitung}) setzen l"asst,
n"amlich
\begin{align}
\frac12\biggl(
\frac1{h_x^2}(u_{i,j-1}-2u_{ij}+u_{i,j+1})
+
\frac1{h_x^2}(u_{i+1,j-1}-2u_{i+1,j}+u_{i+1,j+1})
\biggr)
&=
\frac1{h_t}(u_{i+1,j}-u_{ij})
\label{heat:diskret}
\end{align}
Wir stellen die Gleichung um so, dass der linken Seite nur die 
Variablen f"ur die Zeit $(i+1)h_t$ stehen, und rechts nur die
Varaiblen f"ur die Zeit $ih_t$:
\begin{align}
\frac1{2h_x^2}(u_{i+1,j-1}-2u_{i+1,j}+u_{i+1,j+1})-\frac1{h_t}u_{i+1,j}
&=
-\frac1{2h_x^2}(u_{i,j-1}-2u_{ij}+u_{i,j+1})-\frac1{h_t}u_{ij}
\notag
\\
\frac1{2h_x^2}u_{i+1,j-1}
-\biggl(\frac1{h_x^2}+\frac1{h_t}\biggr)u_{i+1,j}
+\frac1{2h_x^2} u_{i+1,j+1}
&=
-\frac1{2h_x^2}u_{i,j-1}
+\biggl(\frac1{h_x^2}-\frac1{h_t}\biggr)u_{ij}
-\frac1{2h_x^2} u_{i,j+1}
\label{heat:gleichung}
\end{align}
Dies ist ein lineare Gleichung f"ur die Variablen $u_{i+1,j}$.
Die rechte Seite wird berechnet aus den Werten von $u_{ij}$,
f"ur $i=0$ steht dazu die Anfangsfunktion $f$ zur Verf"ugung.

Das Gleichungssystem (\ref{heat:gleichung}) hat die Matrix
\[
A=
\underbrace{
\frac1{2h_x^2}
\begin{pmatrix}
-2& 1& 0&      &  \\
 1&-2& 1&      &  \\
 0& 1&-2&      &  \\
  &  &  &\ddots&  \\
  &  &  &      &-2
\end{pmatrix}}_N
\underbrace{
-\frac1{h_t}
\begin{pmatrix}
 1& 0& 0&      &  \\
 0& 1& 0&      &  \\
 0& 0& 1&      &  \\
  &  &  &\ddots&  \\
  &  &  &      & 1
\end{pmatrix}}_M
=N + M.
\]
Die Matrix ist tridiagonal, und l"asst sich effizient mit dem
Gauss-Algorithmus l"osen.
Dieser Vorteil verschwindet allerdings, sobald das Problem auf
mehr Dimensionen verallgemeinert wird.
Wir wollen daher diesen Ansatz nicht weiterverfolgen, und stattdessen
versuchen, ein iteratives L"osungsverfahren anzuwenden.

Die Matrix $A$ hat eine naheliegende Zerlegung $A=N+M$, so dass
die allgemeine Theorie angewendet werden kann "uber iterative Verfahren,
die zu zerlegbare Matrizen geh"oren.
Schreiben wir $u^{(k)}_{i+}=u^{(k)}_{i+1,j}$ f"ur die $k$-te Iteration f"ur die
L"osung $u_{i+1,j}$ der Gleichung (\ref{heat:gleichung}).
Die allgemeine Theorie besagt, dass der iterative Algorithmus mit
Iterationsformel
\begin{equation}
u^{(k)}_{i+1}=M^{-1}(b-Nu_{i+1}^{(k-1)}),\qquad b=Mu_{i}-Nu_{i},
\label{heat:iteration}
\end{equation}
konvergiert, wenn die Matrix $M^{-1}N$ Spektralradius kleiner als $1$ hat.
Im vorliegenden Fall ist $M=\frac1{h_t}E$ sehr einfach zu invertieren,
es ist $M^{-1}=h_tE$, so dass die fragliche Matrix 
\[
M^{-1}N=\frac{h_t}{2h_x^2}
\begin{pmatrix}
-2& 1& 0&      &  \\
 1&-2& 1&      &  \\
 0& 1&-2&      &  \\
  &  &  &\ddots&  \\
  &  &  &      &-2
\end{pmatrix}
=\frac{h_t}{2h_x^2}\operatorname{tridiag}_n(1,-2,1)
\]
ist.
Der Spektralradius ist
\[
\varrho(M^{-1}N)
=
\varrho\biggl(\frac{h_t}{2h_x^2}\operatorname{tridiag}_n(1,-2,1)\biggr)
=
\frac{h_t}{2h_x^2}\varrho(\operatorname{tridiag}_n(1,-2,1)),
\]
wir haben den folgenden Satz bewiesen:

\begin{satz}
Das Iterationsverfahren (\ref{heat:iteration}) konvergiert, wenn 
\[
\frac{h_t}{2h_x^2}\varrho(\operatorname{tridiag}_n(1,-2,1))<1.
\]
\end{satz}

Wir m"ussen also herausfinden, wie gross der Betrag des
betragsm"assig gr"ossten Eigenwerts
der tridiagonalen Matrix $\operatorname{tridiag}_n(1,-2,1)$ ist.
Diese Frage wird von folgendem Satz beantwortet.

\begin{satz}
\[
\varrho(\operatorname{tridiag}_n(1,-2,1))<4.
\]
\end{satz}

\begin{proof}[Beweis]
Wir berechnen das charakteristische Polynom und zeigen dann, dass es
keine Nullstellen mit Betrag $>4$ hat.
\begin{align*}
\chi_n(\lambda)
&=
\det(\operatorname{tridiag}_n(1,-2-\lambda,1)
=
\left|
\,
\begin{matrix}
-2-\lambda&         1&      &          \\
1         &-2-\lambda&      &          \\
          &          &\ddots&          \\
          &          &      &-2-\lambda
\end{matrix}
\,
\right|
\\
&=(-1)^n
\left|
\,
\begin{matrix}
2+\lambda&        -1&      &         \\
-1        &2+\lambda&      &         \\
          &         &\ddots&         \\
          &         &      &2+\lambda
\end{matrix}
\,
\right|
=(-1)^n p_n(2+\lambda)=(-1)^np_n(\mu)
\end{align*}
Das Polynom $p_n(\mu)$ kann mit dem Entwicklungssatz ermittelt
werden:
\begin{align}
p_n(\mu)
&=
\left|
\,
\begin{matrix}
\mu& -1&   &      &   \\
-1 &\mu& -1&      &   \\
   & -1&\mu&      &   \\
   &   &   &\ddots&   \\
   &   &   &      &\mu
\end{matrix}
\,
\right|
=
\mu
\left|
\,
\begin{matrix}
\mu& -1&      &   \\
 -1&\mu&      &   \\
   &   &\ddots&   \\
   &   &      &\mu
\end{matrix}
\,
\right|
-(
-1)\cdot
\left|
\,
\begin{matrix}
 -1&   &      &   \\
 -1&\mu&      &   \\
   &   &\ddots&   \\
   &   &      &\mu
\end{matrix}
\,
\right|
\notag
\\
&=\mu p_{n-1}(\mu)- p_{n-2}(\mu).
\label{heat:rekursion}
\end{align}
Wir haben also eine Rekursionsgleichung f"ur die Polynome $p_n(\mu)$.
Die Anfangsbedingungen der Rekursion sind
\begin{align*}
p_1(\mu)&=\mu,\\
p_2(\mu)&=\mu^2-1.
\end{align*}
Wir behaupten, dass $p_n(\mu)$ keine Nullstellen mit $|\mu|>2$ hat.
Zun"achst stellen wir fest, dass $|p_2(\mu)| > |p_1(\mu)|$ ist, solange
$|\mu| > 2$, wie man sich durch eine Zeichnung der Graphen von $p_1(\mu)$
und $p_2(\mu)$ "uberzeugen kann.
Wir wollen jetzt mit vollst"andiger Induktion zeigen,
dass ganz allgemein
\begin{equation}
|p_n(\mu)| > |p_{n-1}(\mu)|
\label{heat:pnminorization}
\end{equation}
f"ur alle Werte $|\mu|>2$ und alle $n>1$ gilt. Die Induktionsverankerung
haben wird soeben gegeben.

F"ur den Induktionsschritt d"urfen wir also annehmen,
dass $|p_n(\mu)| > |p_{n-1}(\mu)|$, und m"ussen zeigen,
dass auch $|p_{n+1}(\mu)| > |p_n(\mu)|$ gilt.
Aus der Rekursionsformel (\ref{heat:rekursion})
kann man jetzt folgern:
\begin{align*}
|p_{n+1}(\mu)|
&=|\mu p_{n}(\mu) - p_{n-1}(\mu)|\\
&\ge |\mu|\cdot |p_{n}(\mu)| - |p_{n-1}(\mu)|\\
&>  |\mu|\cdot |p_{n}(\mu)| - |p_{n}(\mu)|\\
&=(|\mu|-1) p_{n}(\mu) > p_{n}(\mu)
\end{align*}
wegen $|\mu|>2$.
Damit ist der Induktionsschritt vollzogen.

Aus (\ref{heat:pnminorization})
kann man aber auch ablesen, dass $p_n(\mu)$ keine Nullstellen
haben kann mit $|\mu| > 2$, jede Nullstellen von $p_n(\mu)$ muss daher
$|\mu|<2$ erf"ullen.

Uns interessieren nat"urlich die Eigenwerte, also die Werte $\lambda=\mu-2$,
f"ur die $p_n(\mu)=0$ ist. Es folgt
\[
|\lambda|=|\mu - 2| \le |\mu|+2 \le 4,
\]
also kann kein Eigenwert einen Betrag $>4$ haben, der Spektralradius ist $4$.
\end{proof}
\begin{satz}
\label{heat:satz-konvergenz}
Das Iterationsverfahren (\ref{heat:iteration}) konvergiert, wenn 
\[
\frac{2h_t}{h_x^2}<1
\qquad
\Leftrightarrow
\qquad
h_t<\frac{h_x^2}2.
\]
\end{satz}
Halbiert man die Gitterkonstante $h_x$ in $x$-Richtung, dann muss man
die Zeitschritte viermal kleiner machen, damit das Verfahren immer
noch konvergent ist.

Die allgemeine Theorie l"asst noch etwas genauere Aussagen "uber die
Konvergenzgeschwindigkeit.
Der Fehler nach $k$ Iterationsschritten ist $|\varrho(M^{-1}N)|^k$.
Wenn die Konvergenz einigermassen schnell sein soll, dann muss
der Spektralradius klein gemacht werden.
Die Anzahl $k$ der Iterationsschritte, um Genauigkeit $\varepsilon< 1$ zu
erreichen ist
\[
k\ge
\log\varepsilon
\log\biggl(\frac{2h_t}{h_x^2}\biggr)^{-1}.
\]

Halbiert man $h_t$, dann wird in jedem Iterationsschritt ein zus"atzliches
Bit Genauigkeit gewonnen.
Wenn man mit der Schrittweite $h_t$ die verlangte Genauigkeit
von 32bit in 32 Schritten erreicht, dann kann man mit
Schrittweite $h_t/2$ diese Genauigkeit in 16 Schritten erreichen.
Nat"urlich muss man jetzt
zwei $h_t$-Schritte durchf"uhren, man hat also nichts gewonnen.

Satz \ref{heat:satz-konvergenz} zeigt, dass mit steigender Aufl"osung
in der Ortskoordinate die Aufl"osung in der Zeitkoordinate "uberprpoprtional
gesteigert werden muss.
Dies ist der numerische Ausdruck daf"ur, dass die W"armeleitungsgleichung
unendlich schnelle Wirkungsausbreitung modelliert.

\section{Parallelisierungsansatz}
Die numerische L"osung des diskretisierten Problems soll jetzt parallelisiert
werden.
Im Abschnitt~\ref{heat:section-diskretisation} wurde dargelegt, dass 
die Werte $u_{ij}$ jeweils nur f"ur einen Wert von $i$ berechnet
werden k"onnen.
Erst wenn die $u_{ij}$ berechnet sind kann man weitergehen und die
$u_{i+1,j}$ berechnen.
Die einzige Parallelisierungsm"oglichkeit steckt also in den der Berechnung
von $u_{ij}$ aus $u_{i-1,j}$ mit Hilfe des Iterationsverfahrens
(\ref{heat:iteration}).

Innerhalb des Iterationsverfahrens ist die einzige
Parallelisierungsm"oglichkeit wieder nur die Anwendung von
(\ref{heat:iteration}).
Die Parallelisierung muss daher den Definitionsbereich in einzelne
Intervalle $J_r=\{j_r,j_r+1,\dots,j_{r+1}-1\}$ von Indizes aufteilen,
in denen die Iteration durchgef"uhrt werden kann. 
F"ur den Wert von $u_{i,j}^{(k)}$ werden die Werte
$u_{i-1,j-1}^{(k-1)}$, $u_{i-1,j}^{(k-1)}$ und $u_{i-1,j+1}^{(k-1)}$ ben"otigt. 
F"ur die inneren Punkte im Interval $J_r$, als $j_r < j < j_{r+1}-1$
sind also alle ben"otigten Indizes in $J_r$, aber f"ur die
Werte $u_{ij_r}$ und $u_{i,j_{r+1}-1}$ sind zus"atzlich Indizes
aus den benachbarten Intervallen $J_{r-1}$ und $J_{r+1}$ n"otig.

Parallelisiert man den Algorithmus mit Hilfe von OpenMPI, reicht es also
nicht, dass die einzelnen Prozesse nur ihren Teil von
$u^{(k)}$ zur Verf"ugung haben, sie brauchen von jedem Nachbarprozess
auch noch genau einen zus"atzlichen Wert.

\section{Implementation}
\subsection{Output}
Die Beispielimplementation schreibt als Output ein NetCDF File mit einem
zweidimensionalen Array der $u(x,t)$ Werte.
Ausserdem werden Attribute hinzugef"ugt, in denen die Schrittweiten
$h_x$ und $h_t$ sowie die Zahl $n$ festgehalten sind.

\subsection{OpenMP}
F"ur kleine Problem kann man die Iteration in einem einzigen Prozess
durchf"uhren, und die Parallelisierung von OpenMP durchf"uhren lassen.

\section{Resultate}
