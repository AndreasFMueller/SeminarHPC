%
% skript.tex -- Skript ueber High Performance Computing
%
% (c) 2012 Prof. Dr. Andreas Mueller, HSR
% $Id: ws-skript.tex,v 1.34 2008/11/02 22:46:16 afm Exp $
%
%\documentclass[a4paper,12pt]{book}
%\documentclass[a4paper]{book}
\documentclass{book}
\usepackage{etex}
\usepackage{geometry}
\geometry{papersize={170mm,240mm},total={140mm,200mm},top=21mm,bindingoffset=10mm}
\usepackage[ngerman]{babel}
\usepackage{times}
\usepackage{amsmath}
\usepackage{amssymb}
\usepackage{amsfonts}
\usepackage{amsthm}
\usepackage{graphicx}
\usepackage{fancyhdr}
\usepackage{textcomp}
\usepackage[all]{xy}
\usepackage{txfonts}
\usepackage{alltt}
\usepackage{verbatim}
\usepackage{paralist}
\usepackage{makeidx}
\usepackage{array}
\usepackage{hyperref}
\usepackage{tikz}
\usepackage{placeins}
\usepackage{subfigure}
\usepackage{csquotes}
\usepackage{float}
%\usetikzlibrary{arrows,decorations.pathmorphing,positioning,fit,petri}
\usetikzlibrary{calc,intersections,through,backgrounds,graphs,positioning,shapes,arrows}
\usetikzlibrary{patterns,decorations.pathreplacing}
%\usetikzlibrary{shapes,snakes,trees}
\usetikzlibrary{decorations.pathreplacing}
%\usetikzlibrary{patterns}
\usepackage{siunitx}
\usepackage{tabularx}
\usetikzlibrary{arrows}
\usepackage{listings}
\lstdefinestyle{Matlab}{
  numbers=left,
  belowcaptionskip=1\baselineskip,
  breaklines=true,
  frame=L,
  xleftmargin=\parindent,
  language=Matlab,
  showstringspaces=false,
  basicstyle=\footnotesize\ttfamily,
  keywordstyle=\bfseries\color{green!40!black},
  commentstyle=\itshape\color{purple!40!black},
  identifierstyle=\color{blue},
  stringstyle=\color{orange},
  numberstyle=\ttfamily\tiny
}
\usepackage{caption}
\usepackage{standalone}
\usepackage[backend=bibtex]{biblatex}
\addbibresource{references.bib}
\addbibresource{uebersicht.bib}
\addbibresource{sample/sample.bib}
\addbibresource{radon/RadonBack.bib}
\addbibresource{kugel/kugel.bib}
\addbibresource{julia/julia.bib}
\addbibresource{apfel/apfelmaennchen.bib}
\addbibresource{mapreduce/map.bib}
\addbibresource{montecarlo/MonteCarlo.bib}
\addbibresource{green/quellen.bib}
\addbibresource{stosswellen/stoss.bib}
\addbibresource{crypto/crypto.bib}
\AtEndDocument{\clearpage\ifodd\value{page}\else\null\clearpage\fi}
\makeindex
\begin{document}
\pagestyle{fancy}
\frontmatter
\newcommand\HRule{\noindent\rule{\linewidth}{1.5pt}}
\begin{titlepage}
\vspace*{\stretch{1}}
\HRule
\vspace*{5pt}
\begin{flushright}
{
\LARGE
Mathematisches Seminar\\
\vspace*{20pt}
\Huge
High Performance Computing}
\end{flushright}
\HRule
\begin{flushright}
\vspace{60pt}
\Large
Leitung: Andreas M"uller\\
\vspace{40pt}
\Large
Dorian Amiet,
Hannes Badertscher,
Danilo Bargen,\\
Marco Bassotti,
Reto Christen,
Gregor Dengler,\\
Felix Hofer,
Fabian Klein,
Flavio La Morea,
Andreas Linggi,\\
Tabea M\'endez,
Daniel Monti,
Lukas Murer,\\
Nicol\'as Rom\'an L"uthold,
Christian Schmid,
Philipp Solenthaler,
Pascal Stump,
Stefan Steiner,
Dario Tr"utsch,
Thomas Ziegler
\end{flushright}
\vspace*{\stretch{2}}
\begin{center}
Hochschule f"ur Technik, Rapperswil, 2014
\end{center}
\end{titlepage}
\hypersetup{
    colorlinks=true,
    linktoc=all,
    linkcolor=blue
}
\newcounter{beispiel}
\newenvironment{beispiele}{
\bgroup\smallskip\parindent0pt\bf Beispiele\egroup

\begin{list}{\arabic{beispiel}.}
  {\usecounter{beispiel}
  \setlength{\labelsep}{5mm}
  \setlength{\rightmargin}{0pt}
}}{\end{list}}
\newcounter{uebungsaufgabe}
% environment fuer uebungsaufgaben
\newenvironment{uebungsaufgaben}{
\begin{list}{\arabic{uebungsaufgabe}.}
  {\usecounter{uebungsaufgabe}
  \setlength{\labelwidth}{2cm}
  \setlength{\leftmargin}{0pt}
  \setlength{\labelsep}{5mm}
  \setlength{\rightmargin}{0pt}
  \setlength{\itemindent}{0pt}
}}{\end{list}\vfill\pagebreak}
\newenvironment{teilaufgaben}{
\begin{enumerate}
\renewcommand{\labelenumi}{\alph{enumi})}
}{\end{enumerate}}
% Loesung
\def\swallow#1{
%nothing
}
\newenvironment{loesung}{%
\begin{proof}[L"osung]%
\renewcommand{\qedsymbol}{$\bigcirc$}
}{\end{proof}}
\newenvironment{diskussion}{}{}
\def\keineloesungen{%
\renewenvironment{loesung}{\swallow\begingroup}{\endgroup}%
\renewenvironment{diskussion}{\swallow\begingroup}{\endgroup}%
}
\newenvironment{beispiel}{%
\begin{proof}[Beispiel]%
\renewcommand{\qedsymbol}{$\bigcirc$}
}{\end{proof}}

\tableofcontents
\newtheorem{satz}{Satz}[chapter]
\newtheorem{hilfssatz}{Hilfssatz}[chapter]
\newtheorem{definition}{Definition}[chapter]
\newtheorem{annahme}{Annahme}[chapter]
\mainmatter
\input linsys.tex
\input vorwort.tex
\part{Grundlagen}
\begin{refsection}
\input einleitung.tex
\input maschinen.tex
\input sprachen.tex
\input daten.tex
\input algorithm.tex
\printbibliography[heading=subbibliography]
\end{refsection}

\part{Anwendungen und Weiterf"uhrende Themen}
\lhead{Anwendungen}
\def\chapterauthor#1{{\large #1}\bigskip\bigskip}
\input anwendungen.tex
%\input sample/sample.tex
\input radon/RadonBack.tex
\input kugel/Kugelsternhaufen.tex
\input julia/julia.tex
\input apfel/apfelmaennchenInput.tex
\input mapreduce/mapreduce.tex
\input montecarlo/MonteCarlo.tex
\input heat/heat.tex
\input potential/potential.tex
\input green/green.tex
\input stosswellen/stosswellen.tex
\input crypto/crypto.tex
\vfill
\pagebreak
\lhead{Index}
\rhead{}
\input skript.ind

\end{document}
