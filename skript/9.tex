Verwenden Sie die Potenzmethode, um den 
Spektralradius der $n\times n$-Matrix
\[
A_n=\begin{pmatrix}
-2& 1&              &              &  &  \\
 1&-2&             1&              &  &  \\
  & 1&            -2&\smash{\ddots}&  &  \\
  &  &\smash{\ddots}&\smash{\ddots}&  &  \\
  &  &              &              &-2& 1\\
  &  &              &              & 1&-2
\end{pmatrix}?
\]
abzusch"atzen
(Alle Matrixelement, die mehr als ein Element von der Diagonalen entfernt
sind, sind 0.). Finden Sie auch den Eigenvektor dazu.

\begin{loesung}
Der dominante Eigenwert bestimmt den Spektralradius von $A_n$, wir bestimmen
den Betrag $\lambda$ des dominanten Eigenwertes ihn mit der Potenzmethode.
Dabei findet man folgende Werte:
\begin{center}
\begin{tabular}{|>{$}r<{$}|>{$}c<{$}|>{$}c<{$}|}
\hline
n&\lambda&\varrho(A_n)\\
\hline
  1&   0.00000 & 0.00000\\
  2&   2.00000 & 2.00000\\
  3&   3.41421 & 3.41421\\
  4&   3.61803 & 3.61803\\
  5&   3.73205 & 3.73205\\
  6&   3.80194 & 3.80194\\
  7&   3.84776 & 3.84776\\
  8&   3.87939 & 3.87939\\
  9&   3.90211 & 3.90211\\
 10&   3.91899 & 3.91899\\
 11&   3.93185 & 3.93185\\
 12&   3.94188 & 3.94188\\
 13&   3.94986 & 3.94986\\
 14&   3.95630 & 3.95630\\
 15&   3.96157 & 3.96157\\
 16&   3.96595 & 3.96595\\
 17&   3.96962 & 3.96962\\
 18&   3.97272 & 3.97272\\
 19&   3.97538 & 3.97538\\
 20&   3.97766 & 3.97766\\
%21&   3.97964 & 3.97964\\
%22&   3.98137 & 3.98137\\
%23&   3.98289 & 3.98289\\
%24&   3.98423 & 3.98423\\
%25&   3.98542 & 3.98542\\
%26&   3.98648 & 3.98648\\
%27&   3.98742 & 3.98742\\
%28&   3.98828 & 3.98828\\
%29&   3.98904 & 3.98904\\
%30&   3.98974 & 3.98974\\
%31&   3.99037 & 3.99037\\
%32&   3.99094 & 3.99094\\
%33&   3.99147 & 3.99147\\
%34&   3.99195 & 3.99195\\
%35&   3.99239 & 3.99239\\
%36&   3.99279 & 3.99279\\
%37&   3.99317 & 3.99317\\
%38&   3.99351 & 3.99351\\
%39&   3.99383 & 3.99383\\
%40&   3.99413 & 3.99413\\
%41&   3.99441 & 3.99441\\
%42&   3.99460 & 3.99466\\
%43&   3.99490 & 3.99490\\
%44&   3.98204 & 3.99513\\
%45&   3.99534 & 3.99534\\
%46&   3.98217 & 3.99553\\
%47&   3.99572 & 3.99572\\
%48&   3.98358 & 3.99589\\
%49&   3.99605 & 3.99605\\
%40&   3.98484 & 3.99621\\
\hline
\end{tabular}
\end{center}
Man kann draus ablesen, dass der Spektralradius immer gr"osser wird,
aber den Wert $4$ wohl nicht "uberschreiten wird.
\end{loesung}

