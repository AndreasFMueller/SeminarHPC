Verwenden Sie die Matrix 
\[
T=\begin{pmatrix}
1&0&0\\
1&1&0\\
0&1&1
\end{pmatrix},
\]
um die symmetrische Matrix
\[
A=\begin{pmatrix}
1&1&0\\
1&1&1\\
0&1&1
\end{pmatrix}
\]
nach der Formel $(T^t)^{-1}AT^{-1}$ zu
pr"akonditionieren.
Verbessert sich dadurch die Konditionszahl gegen"uber der nich konditionierten
Matrix $A$?

\begin{loesung}
Im vorangegangenen Beispiel wurde die Konditionszahl von $A$ als $3+2\sqrt{2}$
berechnet.
Die Inverse von $T$ kann mit dem Gauss-Algorithmus bestimmt werden:
\begin{align*}
\begin{tabular}{|>{$}c<{$}>{$}c<{$}>{$}c<{$}|>{$}c<{$}>{$}c<{$}>{$}c<{$}|}
\hline
1&0&0&1&0&0\\
1&1&0&0&1&0\\
0&1&1&0&0&1\\
\hline
\end{tabular}
&
\rightarrow
\begin{tabular}{|>{$}c<{$}>{$}c<{$}>{$}c<{$}|>{$}c<{$}>{$}c<{$}>{$}c<{$}|}
\hline
1&0&0& 1&0&0\\
0&1&0&-1&1&0\\
0&1&1& 0&0&1\\
\hline
\end{tabular}
\rightarrow
\begin{tabular}{|>{$}c<{$}>{$}c<{$}>{$}c<{$}|>{$}c<{$}>{$}c<{$}>{$}c<{$}|}
\hline
1&0&0& 1& 0&0\\
0&1&0&-1& 1&0\\
0&0&1& 1&-1&1\\
\hline
\end{tabular}
\end{align*}
also
\[
T^{-1}=
\begin{pmatrix}
 1& 0&0\\
-1& 1&0\\
 1&-1&1
\end{pmatrix}
\]
Damit kann jetzt die konditionierte Matrix $A'$ berechnet werden als
\begin{align*}
A'=
(T^{-1})^tAT^{-1}
&=
\begin{pmatrix}
 1&-1& 1\\
 0& 1&-1\\
 0& 0& 1
\end{pmatrix}
\begin{pmatrix}
1&1&0\\
1&1&1\\
0&1&1
\end{pmatrix}
\begin{pmatrix}
 1& 0& 0\\
-1& 1& 0\\
 1&-1& 1
\end{pmatrix}
\\
&=
\begin{pmatrix}
 1&-1& 1\\
 0& 1&-1\\
 0& 0& 1
\end{pmatrix}
\begin{pmatrix}
0&1&0\\
1&0&1\\
0&0&1
\end{pmatrix}
=
\begin{pmatrix}
-1&1&0\\
 1&0&0\\
 0&0&1
\end{pmatrix}.
\end{align*}
F"ur die Konditionszahl sind die Eigenwerte des $2\times 2$-Blockes in der
linken oberen Ecke massgebend. Dieser hat das charakteristische
Polynom
\begin{align*}
\left|\begin{matrix}
-1-\lambda&1\\
         1&-\lambda
\end{matrix}\right|
&=
\lambda(\lambda+1)-1=\lambda^2+\lambda-1
\end{align*}
mit den Nullstellen
\[
\lambda_{1,2}=-\frac12\pm\sqrt{\frac14+1}=\frac{-1\pm\sqrt{5}}2
\]
Die Konditionszahl von $A'$ wird damit
\[
\frac{\sqrt{5}+1}{\sqrt{5}-1}=\frac{(\sqrt{5}+1)^2}{4}
=\frac{5+2\sqrt{5}+1}4=\frac{3+\sqrt{5}}2\simeq2.6180,
\]
eine deutliche Verbesserung der Konditionszahl gegen"uber der nicht 
konditionierten Matrix.
\end{loesung}


