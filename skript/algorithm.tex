\chapter{Algorithmen}
\lhead{Algorithmen}
Nicht jeder Algorithmus ist f"ur High Performance Computing geeignet.
Nur wenn gen"ugend viele Rechenschritte soweit unabh"angig sind, dass sie
parallel auf vielen CPUs ausgef"uhrt werden k"onnen, besteht die
Chance, durch Parallelisierung die Leistung einer massiv parallelen Maschine
ausnutzen zu k"onnen. Dieses Kapitel zeigt einige Beispiel von ungeeigneten
Algorithmen und ihren HPC-tauglichen Gegenst"ucken.

F"ur die Eigenwerte und Eigenvektoren einer $n\times n$-Matrix $A$ 
lernt man in der linearen Algebra den folgendne Algorithmus:
\begin{enumerate}
\item Bestimme das charakteristische Polynom
$\chi_{A}(\lambda)=\det(A-\lambda I)$.
\item Finde die Nullstellen $\lambda_1,\lambda_2,\dots,\lambda_n$ von
$\chi_A(\lambda)$.
\item F"ur jede Nullstelle $\lambda_i$, finde eine nichttriviale L"osung $v_i$
des Gleichungssystems $(A-\lambda_i I)v_i=0$.
\end{enumerate}
Dieser Algorithmus funktioniert sehr gut bei kleinen Matrizen, aber sobald
$n$ gr"osser wird, ist der Algorithmus undurchf"uhrbar.
Nur schon die Berechnung des charakteristischen Polynoms mit Hilfe
der Determinante ist unm"oglich.
Aber auch die Berechnung der Eigenwerte als Nullstellen ist, abgesehen
von den numerischen Problemen dieses Unterfangens, in keiner Weise
parallelisierbar.

Das Eigenwertproblem ist nur eines von vielen praktische wichtigen
Problemen, f"ur welches die Methoden, die man meist als erste lernt,
nicht geeignet sind. Auch die L"osung eines linearen Gleichungssystems
mit dem Gauss-Algorithmus wird unpraktikabel, wenn das Gleichungssystem
sehr gross ist, und wie in Anwendungsproblemen h"aufig der Fall ist,
nur wenige Matrix-Eintr"age von 0 verschieden sind.
Das Ziel dieses Kapitels ist daher, eine Reihe von Algorithmen dazustellen,
welche auch f"ur sehr grosse Probleme der linearen Algebra immer noch
effizient durchf"uhrbar sind.

\section{Iterative Methoden zur L"osung linearer Gleichungssysteme}
\rhead{Iterative Methoden}
In diesem Abschnitt suchen wir Methoden zur L"osung eines linearen
Gleichungssystems $Ax=b$ mit einer $n\times n$-Matrix $A$ und einem
$n$-dimensionalen Vektor $b$.
Die in der Praxis tats"achlich zu l"osenden Gleichungssysteme sind meist 
nicht beliebig, es ist daher auch nicht notwendig, L"osungen f"ur beliebige
Matrizen $A$ zu entwickeln.
In Abschnitt~\ref{section-motivation} betrachten wir daher an einem Beispiel,
wie in den Anwendungen lineare Gleichugnssysteme auftauchen, und versuchen,
die Charakteristiken dieser Gleichungssysteme herauszustellen. Die
nachfolgenden Abschnitte k"ummern sich dann nur noch um Methoden, die
f"ur diese Art von Gleichungsssystemen besonders gut geeignet sind.

\subsection{Motivation\label{section-motivation}}
Partielle Differentialgleichungen sind wichtige Lieferanten von 
linearen Gleichungssystemen in den Anwendungen.
Die Wetterentwicklung, elastische Deformation, Schwingung von
Maschinenteilen oder Geb"auden oder Ausbreitung von Wellen sind
Ph"anomene, die mit partiellen Differentialgleichungen beschrieben
werden k"onnen. In diesem Abschnitt wollen wir zeigen, wie eine
partielle Differentialgleichung zu einem linearen Gleichungssystem
Anlass gibt, und welche Eigenschaften dieses Gleichungssystem hat.

Als Beispiel betrachten wir die partielle Differentialgleichung
\begin{equation}
\Delta u
=\frac{\partial^2u}{\partial x^2}+\frac{\partial^2u}{\partial y^2}
=f(x,y),
\label{motivation-pdgl}
\end{equation}
welche das Potential (elektrisches Potential oder das Potential eines
Gravitationsfeldes) einer Ladungs- oder Massenverteilung mit
Dichte $f(x,y)$ beschreibt.
Gesucht ist eine Funktion $u(x,y)$, welche auf dem Einheitsquadrat
$\Omega = \{(x,y)\,|\, 0\le x,y\le 1\}$ definiert ist und auf dem Rand bestimmte
vorgegebene Werte annimmt:
\begin{equation}
u(x,y)=g(x,y)\qquad x\in\partial \Omega,
\label{motivation-randbedingung}
\end{equation}
wobei $\partial\Omega$ der Rand des Gebietes $\Omega$ ist. Die allgemeine
Theorie der partiellen Differentialgleichungen beweist, dass dieses
Problem eine eindeutige L"osung hat.
Im konkreten Fall eines rechteckingen Gebietes liefert die Theorie sogar
ein L"osungsverfahren, welches mindestens f"ur nicht zu komplizierte
Funktion $f$ und $g$ von Hand durchgef"uhrt werden kann.
F"ur beliebig geformte Gebiete $\Omega$ gibt es jedoch kein analytisches
L"osungsverfahren.

F"ur die numerische L"osung ersetzen wir die auf dem ganzen Quadrat
definierte Funktion $u(x,y)$ durch Zahlenwerte $u_{ij}$, welche nur
auf einem Gitter mit Knoten $(ih, jh)$ mit einer Gitterkonstanten $h = 1/n$
definiert sind.
Die Randbedingungen (\ref{motivation-randbedingung}) f"uhren unmittelbar
auf die Gleichungen
\begin{equation}
u_{ij}=g_{ij}=\begin{cases}
g(0,hj)&\qquad i = 0\\
g(1,hj)&\qquad i = n\\
g(hi,0)&\qquad j = 0\\
g(hi,1)&\qquad j = n
\end{cases}
\label{motivation-randwerte}
\end{equation}
Um die Differentialgleichung (\ref{motivation-pdgl}) durch lineare Gleichungen
zu ersetzen, muss eine Approximation der zweiten Ableitungen
gefunden werden.
Die erste Ableitung in $x$-Richtung kann angen"ahert werden durch
\begin{align*}
\frac{\partial u}{\partial x}(ih,jh)&\simeq \frac{u_{i+1,j}-u_{ij}}h
&
\frac{\partial u}{\partial y}(ih,jh)&\simeq \frac{u_{i,j+1}-u_{ij}}h.
\end{align*}
Daraus kann dann auch eine Approximation von der zweiten Ableitung
gefunden werden:
\begin{align*}
\frac{\partial^2 u}{\partial x^2}(ih,jh)&=\frac1h\biggl(
\frac{u_{i+1,j}-u_{ij}}h-\frac{u_{ij}-u_{i-j,j}}h
\biggr)
=\frac1{h^2}(u_{i-1,j}-2u_{ij}+u_{i+1,j}),\\
\frac{\partial^2 u}{\partial y^2}(ih,jh)&=
=\frac1{h^2}(u_{i,j-1}-2u_{ij}+u_{i,j+1}).
\end{align*}
Die Differentialgleichung wir damit zu einer approximativen linearen
Gleichung 
\begin{equation}
\frac1{h^2}(u_{i-1,j}+u_{i,j-1}+u_{i+1,j}+u_{i,j+1}-4u_{ij})=f_{ij}=f(ih,jh)
\qquad 0<i,j<n.
\label{motivation-lingl}
\end{equation}
Statt der Differentialgleichung (\ref{motivation-pdgl}) mit den Randbedingungen
(\ref{motivation-randbedingung}) ist jetzt also das lineare Gleichungssystem
f"ur die $u_{ij}$ bestehend aus den Gleichungen (\ref{motivation-lingl}) und
(\ref{motivation-randwerte}) zu l"osen.
Die Randwertgleichungen legen einen Teil der $u_{ij}$ bereits fest,
man kann sie zur rechten Seite der Gleichungen schlagen,
so dass nur noch die Werte $u_{ij}$ an inneren Knotenstellen $0<i,j<n$
zu bestimmen.

Wir numerieren wir die verbleibenden Unbekannten neu mit einem
einzigen Index und nennen sie $U_k$,
\begin{equation}
\left.
\begin{aligned}
u_{ij}&=U_{j + (i - 1)(n - 1)}
\\
f_{ij}&=f_{j + (i-1)(n-1)}
\end{aligned}
\right\}
\quad 0<i,j<n.
\end{equation}
Die Gleichungen (\ref{motivation-lingl}) werden jetzt zu Gleichungen
f"ur die $U_k$:
\begin{equation}
U_{k-1}+U_{k+1}+U_{k-(n-1)}+U_{k+(n-1)}-4U_k=f_k.
\end{equation}
Die $(n-1)^2\times(n-1)^2$-Koeffizientenmatrix dieses Gleichungssystems ist
\[
A=\left(
\begin{array}{ccccc|ccccc|c|ccccc}
    -4&     1&     0&\cdots&     0 &     1&     0&     0&\cdots&     0 &\cdots &      &      &      &      &      \\
     1&    -4&     1&\cdots&     0 &     0&     1&     0&\cdots&     0 &\cdots &      &      &      &      &      \\
     0&     1&    -4&\cdots&     0 &     0&     0&     1&\cdots&     0 &\cdots &      &      &     0&      &      \\
\vdots&\vdots&\vdots&\ddots&\vdots &\vdots&\vdots&\vdots&\ddots&\vdots &       &      &      &      &      &      \\
     0&     0&     0&\cdots&    -4 &     0&     0&     0&\dots &     1 &\cdots &      &      &      &      &      \\
\hline
     1&     0&     0&\cdots&     0 &    -4&     1&     0&\dots &     0 &\cdots &      &      &      &      &      \\
     0&     1&     0&\cdots&     0 &     1&    -4&     1&\dots &     0 &\cdots &      &      &      &      &      \\
     0&     0&     1&\cdots&     0 &     0&     1&    -4&\dots &     0 &\cdots &      &      &     0&      &      \\
\vdots&\vdots&\vdots&\ddots&\vdots &\vdots&\vdots&\vdots&\ddots&\vdots &       &      &      &      &      &      \\
     0&     0&     0&\cdots&     1 &     0&     0&     0&\cdots&    -4 &\cdots &      &      &      &      &      \\
\hline
\vdots&\vdots&\vdots&      &\vdots &\vdots&\vdots&\vdots&      &\vdots &\ddots &\vdots&\vdots&\vdots&      &\vdots\\
\hline
      &      &      &      &       &      &      &      &      &       &\cdots &    -4&     1&     0&\cdots&     0\\
      &      &      &      &       &      &      &      &      &       &\cdots &     1&    -4&     1&\cdots&     0\\
      &      &     0&      &       &      &      &     0&      &       &\cdots &     0&     1&    -4&\cdots&     0\\
      &      &      &      &       &      &      &      &      &       &       &\vdots&\vdots&\vdots&\ddots&\vdots\\
      &      &      &      &       &      &      &      &      &       &\cdots &     0&     0&     0&\cdots&    -4\\
\end{array}
\right)
\]
Die Matrix $A$ hat auf der Diagonalen Kopien einer sogenannten tridiagonalen 
$(n-1)\times(n-1)$-Matrix
\[
T=\begin{pmatrix}
    -4&     1&     0& \dots&     0\\
     1&    -4&     1& \dots&     0\\
     0&     1&    -4& \dots&     0\\
\vdots&\vdots&\vdots&\ddots&\vdots\\
     0&     0&     0& \dots&    -4\\
\end{pmatrix}
\]
Mit Hilfe von $T$ kann $A$ in Blockform geschrieben werden:
\[
A=
\begin{pmatrix}
T&E& & &      & \\
E&T&E& &      & \\
 &E&T&E&      & \\
 & &E&T&      & \\
 & & & &\ddots& \\
 & & & &      &T\\
\end{pmatrix}
\]
Alle leeren Felder sind dabei gleich $0$ zu setzen, $E$ bezeichnet
eine $(n-1)\times(n-1)$-Einheitsmatrix.
Die meisten Eintr"age der Matrix $A$ sind also $0$, 
nur maximal 4 Eintr"age in jeder Zeile sind von $0$ verschieden.
Man nennt Matrizen, die vor allem aus $0$-Eintr"agen bestehen,
{\em d"unnbesetz} oder {\em schwachbesetzt}.

Ausserdem sind diese Eintr"age jeweils nicht weit von der Diagonalen
entfernt, die Elemente $1$ in den $E$-Bl"ocken sind genau $n$ Indizes
von der Diagonalen entfernt. Eine solche Matrix nennt man auch eine
Band-Matrix.
Eine $n\times n$ Matrix $B$ heisst \textsl{Bandmatrix} mit Bandbreite $l=p+q+1$,
wenn alle Eintr"age $a_{ij} = 0$ mit $i -j > p+1$ oder $j-i > q+1$. Eine solche
Bandmatrix hat die Form 
\[
B=\begin{pmatrix}
   a_{11}&   a_{12}&\dots &a_{1,q+1}  & 0         &\dots &     0\\
   a_{21}&   a_{22}&\dots &a_{2,q+1}  &a_{2,q+2}  &\dots &     0\\
   \vdots& \vdots  &\ddots&\vdots     &\vdots     &\ddots&\vdots\\
a_{p+1,1}&a_{p-1,2}&\dots &a_{p+1,q+1}&a_{p+1,q+2}&\dots &     0\\
        0&a_{p+2,2}&\dots &a_{p+2,q+1}&a_{p+2,q+2}&\dots &     0\\
   \vdots&\vdots   &\ddots&\vdots     &\vdots     &\ddots&\vdots\\
        0&        0&\dots &     0     &     0     &\dots &a_{nn}\\
\end{pmatrix}.
\]

Dies ist eine charakteristische Eigenschaft linearer Gleichungssysteme,
die aus partiellen Differentialgleichungen abgeleitet wurden.
Unser Hauptaugenmerk in diesem Kapitel m"ussen also Methoden sein,
die f"ur d"unn besetzte Matrizen oder f"ur Bandmatrizen gut funktionieren.

\subsection{Jacobi-Algorithmus}
Wir illustrieren den Jacobi-Algorithmus an folgendem Beispielproblem:
\begin{equation}
A=\begin{pmatrix}1&\frac12\\\frac12&1\end{pmatrix},\quad
b=\begin{pmatrix}1\\-1\end{pmatrix}.
\label{jacobi-beispiel}
\end{equation}
Nat"urlich kann man dieses Gleichungssystem mit dem Gauss-Algorithmus
l"osen, und die L"osung
\[
v=\begin{pmatrix}2\\-2\end{pmatrix}\qquad\Rightarrow\qquad
Av=
\begin{pmatrix}1&\frac12\\\frac12&1\end{pmatrix}
\begin{pmatrix}2\\-2\end{pmatrix}
=
\begin{pmatrix}2-\frac12\cdot 2\\2\frac12-2\end{pmatrix}
=
\begin{pmatrix}1\\-1\end{pmatrix}=b.
\]
finden. Wir k"onnen jede der Gleichungen 
\[
\begin{linsys}{2}
       x_1&+&\frac12x_2&=&1\\
\frac12x_1&+&       x_2&=&-1
\end{linsys}
\]
aber auch nach einer Variablen aufl"osen:
\[
\begin{linsys}{2}
x_1&=& 1&-&\frac12x_2\\
x_2&=&-1&-&\frac12x_1
\end{linsys}
\]
Mit anderen Worten: falls $x_1$ und $x_2$
nicht die L"osung ist, kann mit jeder dieser Gleichungen ein neuer
Wert f"ur eine der Variablen berechnet werden, mit dem wenigstens
eine der Gleichungen stimmen wird.
Wendet man allerdings beide
Gleichungen an, wird sich wieder ein Paar von Werten ergeben,
welche keine L"osung darstellen. Mit etwas Gl"uck k"onnte jedoch
wiederholte Anwendung der Gleichungen immer n"aher an eine 
L"osung kommen. Wir erhalten also eine Folge von Vektoren $v^{(k)}$
\[
v^{(k)} =
\begin{pmatrix}x_1^{(k)}\\x_2^{(k)}\end{pmatrix}
=
\begin{pmatrix}
 1-\frac12x_2^{(k-1)}\\
-1-\frac12x_1^{(k-1)}
\end{pmatrix}
\]
Man findet folgende numerischen Resultate:
\begin{center}
\begin{tabular}{|>{$}r<{$}|>{$}r<{$}>{$}r<{$}|>{$}r<{$}|}
\hline
 k&x_1^{(k)}&x_2^{(k)}&\text{Fehler}\\
\hline
 0& 0.000000& 0.000000&2.000000\\
 1& 1.000000&-1.000000&1.000000\\
 2& 1.500000&-1.500000&0.500000\\
 3& 1.750000&-1.750000&0.250000\\
 4& 1.875000&-1.875000&0.125000\\
 5& 1.937500&-1.937500&0.062500\\
 6& 1.968750&-1.968750&0.031250\\
 7& 1.984375&-1.984375&0.015625\\
 8& 1.992188&-1.992188&0.007812\\
 9& 1.996094&-1.996094&0.003906\\
10& 1.998057&-1.998057&0.002953\\
11& 1.999023&-1.999023&0.000977\\
\hline
\end{tabular}
\end{center}
Tats"achlich konvergiert die Folge gegen die korrekte L"osung.
Genauer: der Fehler wird in jedem Iterationsschritt halbiert,
man gewinnt also ein Bit Genauigkeit in jedem Schritt. F"ur dieses
spezielle Gleichungssystem w"ahren also etwa dreissig Iterationsschritte
n"otig, um eine L"osung mit einer Genauigkeit von 10 Stellen zu erhalten.

Allgemein kann man in jedem Gleichungssystem $Ax=b$ die Gleichung $i$
nach der Unbekannten $x_i$ aufl"osen, und ein System von
Iterationsgleichungen
\begin{equation}
x_i^{(k)}=\frac1{a_{ii}}\biggl(b_i
-\sum_{j=1}^{i-1}a_{ij}x_j^{(k-1)}
-\sum_{j=i+1}^na_{ij}x_j^{(k-1)}
\biggr)
\label{jacobi-iteration}
\end{equation}
finden.
Ausgehend vom Nullvektor wird die Iteration unter gewissen
Voraussetzungen, die in Abschnitt \ref{section-konvergenz}
diskutiert werden, gegen eine L"osung konvergieren. Dieses
Verfahren heisst das Jacobi-Verfahren.

Wie gross ist der Rechenaufwand f"ur einen einzelnen Iterationsschritt?
Offenbar h"angt dies von der Anzahl der nicht verschwindenen Elemente 
in jeder Zeile der Matrix $A$ ab.
Ist $l$ die maximale Anzahl nicht verschwindender Elemente in einer Zeile,
dann kann ein Iterationsschritt mit maximal $2ln$ Gleitkommaoperationen
durchgef"uhrt werden. Nehmen wir an, dass der Fehler "ahnlich wie im Beispiel
exponentiell abnimmt, also der Restfehler nach $k$ Iterationen von der
Gr"ossenordnung $q^k$ ist mit $q<1$.
Um die Genauigkeit $\delta$ zu erreichen, sind so viele Iterationen notwendig,
dass $q^k < \delta$ gilt, oder $k>\log \delta/\log q$.
Der Rechenaufwand ist also $2nl\log\delta/\log q$.
Der Jacobi-Algorithmus hat also bedeutend schneller als die $n^3$ 
$n$-Operationen, die f"ur den Gauss-Algorithmus erforderlich sind.
Ausserdem sind die Rechenoperationen des Jacobi-Algorithmus hochgradig
parallelisierbar.

\subsection{Gauss-Seidel}
Im Jacobi-Verfahren wurde aus einer angen"aherten L"osung $v^{(k)}$
eine verbesserte angen"aherte L"osung $v^{(k+1)}$ berechnet.
Diese neuen Werte sind aber auch keine L"osungen, sondern nur verbesserte
Approximationen.
Warum also nicht bei der Berechnung der neuen Approximation $x_j^{(k)}$
nicht nur die Approximationen $x_i^{(k-1)}$ verwenden, sondern soweit
diese schon bestimmt sind, bereits die $x_j^{(k)}$.
Die Iterationsformeln (\ref{jacobi-iteration}) werden dann zu
\begin{equation}
x_i^{(k)}=\frac1{a_{ii}}\biggl(b_i
-\sum_{j=1}^{i-1}a_{ij}x_j^{(k)}
-\sum_{j=i+1}^na_{ij}x_j^{(k-1)}
\biggr)
\label{gauss-seidel-iteration}
\end{equation}
Da in jeder Berechnung die aktuellsten verf"ugbaren Approximationen der
L"osung verwendet werden, kann man erwarten, dass dieses Verfahren
noch etwas schneller konvergiert als das Jacobi-Verfahren.
Es heisst das Gauss-Seidel-Verfahren.

Allerdings ist dieses Verfahren nicht mehr im gleichen Mass parallelisierbar.
Die Variable $x_i^{(k)}$ kann erst berechnet werden, wenn $x_{i-1}^{(k)}$
berechnet ist. Die Gesamtzahl der Operationen ist zwar wegen der erwartet
schnelleren Konvergenz kleiner geworden, aber die Parallelisierbarkeit
beschr"ankt sich auf die Operationen in einer einzelnen Zeile, also
bei der Berchnung der beiden Summen in (\ref{gauss-seidel-iteration}).

Es stellt sich die Frage, ob man die Konvergenz beschleunigen kann,
ohne die Parallelisierbarkeit zu kompromittieren.
Der Gauss-Seidel-Algorithmus zeigt, dass eine Konvergenzbeschleunigung
m"oglich ist, man m"usste aber vom Jacobi-Algorithmus ausgehenG


\subsection{Matrix-Aufteilung}
Das Jacobi- und das Gauss-Seidel-Verfahren sind beide Spezialf"alle
der folgenden Aufteilung der Matrix $A$ in zwei Teile. 
Von einer Matrix-Aufteilung sprechen wir, wenn $A$ als $M+N$ geschrieben
werden kann, so dass
\begin{equation}
Mx=b-Nx
\qquad
\Rightarrow
\qquad
x=M^{-1}(b-Nx).
\end{equation}
Der Jacobi-Algorithmus entspricht dem Fall 
\begin{equation}
M=\begin{pmatrix}
a_{11}&     0&\dots &0\\
     0&a_{22}&\dots &0\\
\vdots&\vdots&\ddots&\vdots\\
     0&     0&\dots &a_{nn}
\end{pmatrix}=D.
\label{jacobi-aufteilung}
\end{equation}
Der Gauss-Seidel-Algorithmus entspricht dem Fall
\[
M=\begin{pmatrix}
a_{11}&     0& \dots&     0\\
a_{21}&a_{22}& \dots&     0\\
\vdots&\vdots&\ddots&\vdots\\
a_{n1}&a_{n2}&\dots &a_{nn}
\end{pmatrix},
\qquad
N=\begin{pmatrix}
     0&a_{12}&\dots &a_{1n}\\
     0&     0&\dots &a_{2n}\\
\vdots&\vdots&\ddots&\vdots\\
     0&     0&\dots &     0
\end{pmatrix}.
\]
Nat"urlich ist so eine Aufteilung nur dann sinnvoll, wenn 
$M$ auf effiziente Art invertierbar ist.
F"ur (\ref{jacobi-aufteilung}) ist dies klar, die Inverse der
Diagonalmatrix $D$ ist die Diagnalmatrix aus den invertierten
Diagonalelemente von $D$.
Im Gauss-Seidel-Fall ist eine Dreicksmatrix zu invertieren, 
was durch die Formeln (\ref{gauss-seidel-iteration}) bereits
erledigt ist.

\subsection{Konvergenz\label{section-konvergenz}}
Die Aufteilung der Matrix $A=M+N$ erm"oglicht nun auch eine Untersuchung
der Konvergenz dieser Verfahren.
Sei $x_0$ die L"osung des Gleichungssystems, dann ist $\delta_k=x^{(k)}-x_0$ der Fehler
nach $k$ Iterationen.
Aus $x^{(k)} = M^{-1}(b-Nx^{(k-1)})$ und $Mx_0 =b-Nx_0$ folgt
\begin{align*}
x^{(k)}-x_0
&=M^{-1}(b-Nx^{(k-1)})-x_0\\
&=M^{-1}(Mx_0+Nx_0-Nx^{(k-1)})\\
&=-M^{-1}N(x^{(k-1)}-x_0)\\
\Rightarrow\qquad \delta_k&=-M^{-1}N\delta_{k-1}.
\end{align*}
Das Verfahren ist also genau dann konvergent, wenn die Matrix
$C=M^{-1}N$, wiederholt auf den initialen Fehler $b$ angewendet,
zu einer gegen Null strebenden Vektorfolge f"uhrt.

Die Zerlegung der Matrix im Beispiel (\ref{jacobi-beispiel}) ist
\[
M=\begin{pmatrix}1&0\\0&1\end{pmatrix},\quad
N=\begin{pmatrix}0&\frac12\\\frac12&0\end{pmatrix}\Rightarrow
C=M^{-1}N=N.
\]
Die Potenzen von $N$ sind 
\[
N^k=\begin{cases}
{\displaystyle\frac1{2^k}}
\begin{pmatrix}0&1\\1&0\end{pmatrix}
&\qquad \text{$k$ ungerade,}\\
{\displaystyle \frac1{2^{k+1}}}E&\qquad \text{$k$ gerade.}
\end{cases}
\]
Damit kann man die L"ange von $\delta_k$ berechen:
\[
|\delta_k|=\frac{|\delta_0|}{2^k}.
\]
Insbesondere wird der Fehler in jedem Iterationsschritt halbiert.

Im allgemeinen Fall braucht man Informationen dar"uber, ob und wie
schnell die Vektoren $C^k\delta_0$ abnehmen.
Nehmen wir an, die Matrix $C$ sei diagonalisierbar, dann gibt es
eine Eigenvektorbasis $u_1,\dots,u_n$ mit Eigenwerten
$\lambda_1,\dots,\lambda_n$. Der Vektor $\delta_0$ kann dann in der
Eigenvektorbasis dargestellt werden als
\[
\delta_0=d_1u_1+\dots+d_nu_n.
\]
Die Fehler der Iterationen werden dann:
\[
\delta_k
=
C^k\delta_0
=
d_1C^ku_1+\dots+d_nC^ku_n
=
d_1\lambda_1^ku_1+\dots+d_n\lambda_n^ku_n.
\]
Die Summe auf der rechten Seite kann nur dann gegen 0 konvergieren, wenn
jeder der $\lambda_i$ Betrag $|\lambda_i|<1$ haben. 

\begin{definition}
Sind $\lambda_i$ alle Eigenwerte einer Matrix $C$, dann heisst
$\varrho(C)=\max_{1\le i\le n}|\lambda_i|$ der Spektralradius
von $C$.
\end{definition}

\begin{satz}
Das auf der Matrix-Aufteilung $A=M+N$ basierende Iterationsverfahren
konvergiert, wenn $\varrho(M^{-1}N)<1$ gilt.
\end{satz}

Mit dem Spektralradius kann man nun auch berechnen, wieviele Iterationsschritte
f"ur eine bestimmte Genauigkeit notwendig sind.
Dazu nehmen wir an, dass $\lambda_1$ der betragsm"assig gr"osste
Eigenwert ist, mit $|\lambda_1=\varrho(M^{-1}N)=\varrho$.
Dann ist der Fehler nach $k$ Iterationsschritten
\[
|C^k\delta_0|\le \varrho^k|\delta_0|
\]
F"ur die Konvergenzgeschwindigkeit ist ausschlaggebend, wieviel kleiner
als $1$ der Spektralradius von $M^{-1}N$ ist.
Um eine Genauigkeit $\varepsilon$ zu erreichen, muss $k$ so gross
sein, dass
$\varrho^k|\delta_0|=\varrho^k|b|\le \varepsilon$ ist:
\[
k\log\varrho+\log|b|\le \log\varepsilon
\quad
\Rightarrow
\quad
k\ge \frac{\log\varepsilon-\log|b|}{\log\varrho}.
\]

\begin{beispiel}
Kann das Gleichungssystem mit der Matrix 
\[
A=\begin{pmatrix}1&2\\2&1\end{pmatrix}
\]
mit dem Jacobi-Verfahren oder dem Gauss-Seidel-Verfahren gel"ost werden?

Um diese Frage zu entscheiden, muss der Spektralradius der
Matrix $M^{-1}N$ bestimmt werden.
F"ur das Jacobi-Verfahren ist dies die Matrix
\[
M=E,\quad = N\begin{pmatrix}0&2\\2&0\end{pmatrix}\quad
\Rightarrow
\quad
M^{-1}N=N.
\]
Das charakteristische Polynom von $N$ ist $\lambda^2-4$ mit den
Nullstellen $\pm2$, der Spektralradius von $M^{-1}N$ ist also $2> 1$,
das Verfahren ist also nicht konvergent.

F"ur das Gauss-Seidel-Verfahren findet man f"ur $M^{-1}N$
\[
M=\begin{pmatrix}1&0\\2&1\end{pmatrix},\quad
N=\begin{pmatrix}0&2\\0&0\end{pmatrix}\quad
\Rightarrow
\quad
M^{-1}N=\begin{pmatrix}0&2\\0&-4\end{pmatrix}
\]
mit charakteristischem Polynom $\lambda(\lambda+4)$, mit Nullstellen
$0$ und $-4$. Der Spektralradius ist also $4$, das Verfahren kann nicht
konvergieren.
\end{beispiel}


%\subsection{Konjugierte Gradienten}

\section{Iterative Methoden zur Eigenwertbestimmung}
\rhead{Eigenwerte}
Das Eigenwertproblem ist von zentraler Bedeutung in Ingenieur-Anwendungen,
doch das Verfahren, welches man in Anf"angervorlesungen lernt, ist f"ur die
numerische Berechnung ungeeignet.
Wir d"urfen wieder davon ausgehen, dass die Matrizen d"unn besetzt sind, oder
sogar Bandmatrizen sind.
Im Falle einer Matrix, die von einer partiellen Differentialgleichung
zweiter Ordnung herr"uhrt, k"onnen wir sogar annehmen, dass die
Matrix symmetrisch ist.

Wir werden dabei die Lehren aus dem letzten Abschnitt ziehen, wo sich
gezeigt hat, dass eine iterative L"osung zwar mehr Schritte ben"otigt,
daf"ur aber besser parallelisierbar ist, so dass sich insgesamt trotzdem
eine effiziente L"osung ergibt.

\subsection{Potenzmethode}
Sei $A$ eine $n\times n$-Matrix und sei $\lambda_1$ der betragsm"assig
gr"osste Eigenwert von $A$, und $v_1$ der zugeh"orige Eigenvektor.
Wir nehmen weiter an, dass alle anderen Eigenwerte $\lambda_i$ zugeh"orige
Eigenvektoren $v_i$ haben sollen. Ein beliebiger $n$-dimensionaler Vektor $u$
kann jetzt in der Eigenvektorbasis mit Koeffizienten $a_i$ als
\[
u=
a_1v_1+a_2v_2+\dots+a_nv_n
\]
dargestellt werden.
Wendet man darauf wiederholt die Matrix $A$ an, ergibt sich daraus
\[
A^ku=
a_1\lambda_1^kv_1
+
a_2\lambda_2^kv_2
+
\dots
+
a_n\lambda_n^kv_n.
\]
Wenn $|\lambda_1| > 1$ ist, werden die Potenzen $\lambda_i^k$ "uber
alle Grenzen wachsen.
Ist $|\lambda_i|<1$, werden die Potenzen beliebig klein.
In beiden F"allen werden die Vektoren $A^ku$ f"ur grosse Werte von $k$
nicht weiter n"utzlich sein.
Wenn man aus den $A^ku$ irgend etwas herausholen will, muss man $A^ku$ 
immer wieder auf eine vern"unftige Gr"osse zur"uck holen, zum Beispiel,
indem man durch $\lambda_1^k$ teilt:
\begin{equation}
\frac1{\lambda_1^k}A^ku=a_1u_1
+\biggl(\frac{\lambda_2}{\lambda_1}\biggr)^ku_2
+\dots+
\biggl(\frac{\lambda_n}{\lambda_1}\biggr)^ku_n
\label{potenzverfahren}
\end{equation}
Da alle Eigenwerte $|\lambda_i|<|\lambda_1|$ f"ur jedes $i>1$ gilt,
sind alle Quotienten auf der rechten Seite von (\ref{potenzverfahren})
kleiner als eins, und gehen daher mit $k\to\infty$ gegen Null. Man kann
also mit Hilfe der Iteration
\[
\lim_{k\to\infty}\frac1{\lambda_1^k}A^ku=a_1u_1
\]
den Eigenvektor zum betragsgr"ossten Eigenwert bekommen.

Meistens wird man aber den Wert des betragsgr"ossten Eigenwerts
gar nicht kennen, die Renormierung in (\ref{potenzverfahren}) 
ist also gar nicht m"oglich. Allerdings ist es auch nicht n"otig, 
die Renormierung mit $\lambda_1^{-k}$ durchzuf"uhren. Das Ziel
der Renormierung ist ja nur, den Vektor $A^ku$ in vern"unftiger
gr"osse zu behalten. Es reicht also, nach jeder Anwendung von $A$ 
den Vektor wieder auf L"ange $1$ zu bringen.
Auch in diesem Fall werden die Komponenten $u_2,\dots,u_n$, 
immer kleiner, im Grenzwert $k\to\infty$ bleibt nur die Komponente
$u_1$ normiert auf L"ange $1$.

Der neue Algorithmus lautet also: man beginne mit irgen einem Vektor
$u_0$, und konstruiere daraus die Folge
\begin{equation}
u_{k+1}=\frac{Au_k}{|Au_k|}, \quad k>1.
\label{potenz-iteration}
\end{equation}

\begin{beispiel}
Bestimme den dominanten Eigenwert und den zugeh"origen Eigenvektor
der Matrix
\[
A=\begin{pmatrix}-2&1&0\\1&-2&1\\0&1&-2\end{pmatrix}.
\]

Wir beginnen mit dem Vektor $u_0 =(1, 1, 1)^t$ und wenden die Iteration
(\ref{potenz-iteration}) widerholt darauf an. Dabei erhalten wir die
Vektoren
\begin{center}
\begin{tabular}{|>{$}r<{$}|>{$}r<{$}>{$}r<{$}>{$}r<{$}|}
\hline
 k&u_{k,1} &u_{k,2} &u_{k,3}\\
\hline
 0& 1.00000& 0.00000& 0.00000\\
 1&-0.70711& 0.00000&-0.70711\\
 2& 0.57735&-0.57735& 0.57735\\
 3& 0.51450& 0.68599&-0.51450\\
 4& 0.50252&-0.70353& 0.50252\\
 5&-0.50043& 0.70649&-0.50043\\
 6& 0.50007&-0.70700& 0.50007\\
 7&-0.50001& 0.70709&-0.50001\\
 8& 0.50000&-0.70710& 0.50000\\
 9&-0.50000& 0.70711&-0.50000\\
\hline
\end{tabular}
\end{center}
Offenbar konvergiert die Vektorfolge gegen den Vektor
\[
u_1=\begin{pmatrix}-0.50000\\0.70711\\-0.50000\end{pmatrix}.
\]
Allerdings wechselt der Vektor in jedem Schritt das Vorzeichen,
was nat"urlich darauf zur"uckzuf"uhren ist, dass $\lambda_1<0$.
Da der Vektor jetzt bekannt ist, kann man durch Vergleich von
$u_1$ mit $Au_1$ auch den Eigenwert finden, es ist $\lambda_1=-3.4142$.
\end{beispiel}

\subsection{Weitere Eigenvektoren}
Um die "ubrigen Eigenvektoren zu finden, muss man sich jetzt nur noch
auf den von den Eigenvektoren $u_2,\dots,u_n$ aufgespannten Raum
beschr"anken.
Das Problem ist allerdings, dass man diesen Raum gar nicht kennt,
man hat die Eigenvektoren ja noch gar nicht bestimmt.
Wenn allerdings die Matrix $A$ symmtreisch ist, dann weiss man,
dass die Eigenvektoren $u_2,\dots,u_n$ senkrecht stehen auf dem
bereits gefunden Vektor $u_1$.
Man muss also nur eine Basis von Vektoren senkrecht auf $u_1$ finden,
was man mit dem Gram-Schmidtschen Orthogonalisierungsverfahren 
erreicht werden kann.
In dieser neuen Basis wird die Matrix $A$ die Form
\[
A=
\left(
\begin{array}{c|ccc}
\lambda_1&0&\dots&0\\
\hline 
	0& &     & \\
  \vdots & &  A_1& \\
	0& &     & 
\end{array}
\right)
\]
Die Matrix $A_1$ ist wieder eine symmetrische $(n-1)\times(n-1)$-Matrix,
die die Eigenwerte $\lambda_2,\dots,\lambda_n$.

Dieses Verfahren ist offenbar etwas umst"andlich, und nicht unbedingt
in dieser Form durchf"uhrbar.
Das Potenzverfahren kann jedoch in einer Art formuliert werden, dass
es mittels Iteration alle Eigenwerte und Eigenvektoren einer beliebigen
Matrix bestimmen kann, selbst wenn die Eigenvektoren komplex sind.
Dieses erweiterte Potenzverfahren heisst Francis Verfahren, es
wurde von John Francis 1961 entwickelt.
Eine gute Beschreibung findet man auch im Buch von Watkins \cite{watkins}.

