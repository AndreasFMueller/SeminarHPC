\chapter{Umgang  mit der Datenflut}
\lhead{Umgang mit der Datenflut}
\rhead{Kennzahlen}
High Performance Computing produziert mit h"ochstm"oglicher Geschwindigkeit
gr"osstm"ogliche Datenmengen. Diese Daten m"ussen gespeichert, verarbeitet
und visualisiert werden.
Dieses Kapitel diskutiert einige Techniken, wie man in einem HPC-Problem
mit der Datenflut umgehen kann.

\section{Kennzahlen\label{section-kennzahlen}}
Eine Str"omungssimulation produziert f"ur jeden Punkt des Raumes und
f"ur jeden Zeitpunkt Werte f"ur Druck, Temperatur und Geschwindigkeit.
Selbst f"ur ein kleines Problem entsteht so eine Flut von Daten, die
in ihrer Gesamtheit kaum n"utzlich sind. 

F"ur die Anwendung ist selten die Gesamtheit der Daten n"utzlich.
Bei der Str"omungssimulation sind Dinge wie Auftrieb, Widerstandskoeffizient,
Temperaturen an kritischen Stellen viel aussagekr"aftiger als die Details
des Str"omungsfeldes. 

\begin{beispiel}
Als Beispiel betrachten wir eine Menge von Kugeln, die sich auf einem
Billiardtisch reibungslos bewegen k"onnen und miteinander in elastischen
St"ossen interagieren k"onnen. Es ist recht einfach, aus Position und
Gewschwindigkeit zweier
Billiardkugeln den Zeitpunkt des n"achsten Zusammenstosses zu berechnen,
falls dieser "uberhaupt vor dem Zusammenstoss mit der Bande stattfindet.
Auf diese Weise kann man zu jeder belieibigen Zeit die Geschwindigkeit
$v_i(t)$ der Billiardkugel $i$ angeben.

Die Werte $v_i(t)$ sind kaum von Interesse, es l"asst sich daraus nichts
Interessantes ablesen.
Ausserdem "andern diese Zahlen mit jedem Zusammenstoss, die schnellste
Kugel wird nach wenigen Zusammenst"ossen nicht mehr die schnellste sein,
und auch die langsamste ist nicht f"ur alle Zeiten zu dieser Position
verdammt.
Um geeignete Kennzahlen zu finden, die das Verhalten der Gesamtheit
der Vektoren besser beschreiben, k"onnen wir uns an der Physik orientieren,
die mit diesem Modell simuliert wird.
Ein ideales Gas kann man sich als eine Menge von Kugeln vorstellen,
die elastische St"osse ausf"uhren.
Normalerweise beschreibt man ein Gas nicht mit Hilfe der Geschwindigkeit
der einzelnen Molek"ule, sondern man verwendet makroskopische Gr"ossen
wie Temperatur, Druck oder Dichte.

Die Resultate der Simulation m"ussen also noch aufgearbeitet werden, um
solche makroskopische Gr"ossen zu bestimmen.
Zum Beispiel ist der Druck der mittlere Impuls, der in einer Zeiteinheit
auf die Wand "ubertragen wird.
Ein einfache Summe "uber die Normalkomponenten der Geschwindigkeiten
derjenigen Teilchen, die in einem Zeitinterval mit der Wand zusammenstossen,
wir die Frage beantworten.

Die Temperatur ist jedoch etwas schwieriger.
Grunds"atzlich ist die Temperatur im wesentlichen die mittlere 
kinetische Enerige der Molek"ule des Gases. Die Boltzmann-Konstante
$k_B\simeq 1.38065\cdot10^{-23} \text{J/K}$ ist der Proportionalit"atsfaktor
zwischen Energie und Temperatur.
Es reicht aber nicht,
nur die mittlere Geschwindigkeit auszurechnen.
Ein Strahl von Teilchen, die alle die gleiche Geschwindigkeit haben,
verh"alt sich v"ollig anders als ein ideales Gas.
Die oberfl"achliche Definition der Temperatur als der mittleren
kinetischen Energie macht keine Aussage dar"uber, ob sich die
simulierten Dinge auch so verhalten, wie wir das von einem
Gas einer bestimmten Temperatur erwarten.

In einem idealen Gas der Temperatur $T$ haben die Teilchen nicht nur
eine bestimmte mittlere kinetische Energie, die kinetische Energie
oder der Geschwindigkeitsbetrag
der einzelnen Teilchen hat eine ganz bestimmt Verteilung.
Die einzelnen Geschwindigkeitskomponenten entstehen durch Zusammenwirken
vieler einzelner St"osse, nach dem zentralen Grenzwertsatz darf man also
annehmen, dass die Geschwindigkeitskomponenten normalverteilt sind.
Sofern sich das Gas nicht bewegt, ist der Erwartungswert der Komponenten
0, die Geschwindigkeitsvektoren haben also eine Verteilung mit
Wahrscheinlichkeitsdichte
\[
\frac1{(\sqrt{2\pi}\sigma)^3}e^{-\frac{v^2}{2\sigma^2}}.
\]
Die Wahrscheinlichkeit eines Geschwindigkeitsbetrages zwischen $v$ und
$v+\Delta v$ ist also
\[
\int_v^{v+\Delta v} 4\pi v^2
\frac1{(\sqrt{2\pi}\sigma)^3}e^{-\frac{v^2}{2\sigma^2}}\,dv,
\]
also hat die Geschwindigkeitsverteilung die Wahrscheinlichkeitsdichte
\[
4\pi
\biggl(
\frac{1}{2\pi \sigma^2}
\biggr)^{\frac32}
v^2 e^{-\frac{v^2}{2\sigma^2}}.
\]
Man kann den Erwartungswert berechnen, und damit die Konstante $\sigma$
identifizieren, es ist $\sigma^2=k_BT/m$, damit wird die 
Wahrscheinlichkeitsdichte der Maxwell-Boltzmann-Verteilung
\[
p(v)=
4\pi\biggl(\frac{m}{2\pi k_BT}\biggr)^\frac32 v^2 e^{-\frac{mv^2}{2k_BT}}.
\]
Die Auswertung der Simulationsdaten sollte also nicht nur die mittlere
kinetische Energie bestimmen, sondern auch "uberpr"ufen, ob die Verteilung
der Geschwindigkeiten der Boltzmann-Verteilung folgt, was zum Beispiel mit
Hilfe eines Kolmogorov-Smirnov-Tests gemacht werden kann.
\end{beispiel}

In "ahnlicher Art k"onnen in Simulationen oft Kennzahlen gefunden werden,
die eine Beurteilung der Simulationsresultate erm"oglichen, die sich
Plotten lassen, oder die ganz konkreten Gr"ossen in der simulierten
Realit"at entsprechen, und damit die direkte Anwendung der
Simulationsresultate erm"oglichen.

\rhead{Datenspeicherung}
\section{Datenspeicherung}
Die Resultate einer Simulation k"onnen oft nicht vollst"andig gespeichert
werden, die Datenmenge ist m"oglicherweise schlicht zu gross.
Manchmal ist auch nach Anwendung geeigneter Reduktionsmethoden wie
in Abschnitt~\ref{section-kennzahlen} trotz eine substantielle Datenmenge zu
speichern, und es m"ussen Wegen gefunden werden, dass diese Daten
effizient weiterverarbeit werden k"onnen.

\subsection{Datenbanken}
Datenbanken k"onnen grosse Mengen von Daten speichern und insbesondere
beliebige Auswahlen davon schnell zur"uckgeben.
Im Allgemeinen sind relationale Datenbanken jedoch relativ 
schwerf"allig zu handhaben, so dass die direkte Speicherung
der Daten in Files einfacher ist.

Eine Datenbank wird jedoch dann interessant, wenn man nicht nur einen
einzelnen Simulationslauf, sondern eine ganze Menge von Simulationsresultaten
mit allen relevanten Parametern und daraus abgeleiteten Kennzahlen
in einer Datenbank ablegen kann.
Dann entsteht einer Art ``Data warehouse'' f"ur diese Anwendung,
und man kann Datenbank-Werkzeuge verwenden, um die Daten auf
verschiedene Arten zu analysieren. 
Insbesondere kann es interessant sein, auf die gleiche Datenbank
mit einem Werkzeug f"ur statistische Datenanalysen wir R zuzugreifen,
und damit statistische Auswertungen wie im letzten Abschnitt
angedeutet reproduzierbar und mit hoher Zuverl"assigkeit durchf"uhren
zu k"onnen.

\subsection{Fileformate}
Wie speichert man die Daten in Files, wenn doch einmal alle Resultate
gespeichert werden m"ussen?
Nat"urlich kann man sich nach Bedarf irgendwelche Datenformate zurechtlegen,
das bedeutet aber auch, dass man geeignete Software f"ur die
Weiterverarbeitung selber schreiben muss, was selten effizient ist.
Ausserdem sollten die Datenformate plattformunabh"angig sein,
sodass die Weiterarbeitung auf Maschinen beliebiger Architektur
oder in einer beliebigen Programmiersprache erfolgen kann.

Oft sind HPC Resultatdaten mehrdimensional, ein geeignetes Speicherformat
sollte also auch zwei- oder dreidimensionale Datenfelder abspeichern
k"onnen.
Die zugeh"origen Koordinatensysteme sollten ebenfalls im File
spezifiziert werden k"onnen.

F"ur dieses Problem wurden verschiedene Datenformate geschaffen, 
wovon im folgenden drei kurz besprochen werden sollen.

\subsubsection{NetCDF}
\index{NetCDF}
Wikipedia beschreibt NetCDF wie folgt:
\begin{quote}
Network Common Data Format (NetCDF) ist ein Dateiformat f"ur den
Austausch wissenschaftlicher Daten. Es handelt sich um ein bin"ares
Dateiformat, das durch die Angabe der Byte-Reihenfolge im Header
maschinenunabh"angig ist. NetCDF ist ein offener Standard; das Projekt
wird von der University Corporation for Atmospheric Research (UCAR)
betreut.
\end{quote}
Auf der UCAR-Website \url{http://www.unidata.ucar.edu/software/netcdf/}
findet man Sourcecode f"ur Bibliotheken in C, C++, Fortran und Java,
Binaries und Dokumentation. 

\begin{beispiel}
Im Verzeichnis {\tt data} des Sourcecode-Repository befindet sich
ein Beispiel, wie NetCDF geschrieben und gelesen werden k"onnen.

Ein NetCDF File wird mit der Funktion \verb+nc_create+ angelegt:
\begin{verbatim}
if (NC_NOERR != (status = nc_create(filename, NC_NOCLOBBER, &fileid))) {
        fprintf(stderr, "cannot create file %s: %s\n",
                filename, nc_strerror(status));
        ...
}
\end{verbatim}
Das Flag \verb+NC_NOCLOBBER+ bedeutet, dass ein eventuell schon vorhandenes
Datenfile nicht "uberschrieben werden soll.
Dies ist eine sinnvolle Vorsichtsmassnahme, um nicht versehentlich
die wertvollen Output-Daten einer aufwendigen Simulation zu vernichten.

Sobald ein File erstellt worden ist, befindet es sich im Definition
Mode, in dem Variablen und Dimensionen definiert werden k"onnen.
Um zum Beispiel einen dreidimensionalen Array namens {\tt results}
mit Dimensionen {\tt x} der L"ange 140, {\tt y} der L"ange 100 und 
{\tt t} unbegrenzter L"ange zu definieren, sind folgende Befehle n"otig:
\begin{verbatim}
// define dimensions of the array
size_t  lenx = 140, leny = 100;
int     x_dim, y_dim, t_dim;
if (NC_NOERR != (status = nc_def_dim(fileid, "x", lenx, &x_dim))) {
        ...
}
if (NC_NOERR != (status = nc_def_dim(fileid, "y", leny, &y_dim))) {
        ...
}
if (NC_NOERR != (status = nc_def_dim(fileid, "t", NC_UNLIMITED, &t_dim))) {
        ...
}

// define the array
int     arrayid;
int     dimensions[3] = { t_dim, x_dim, y_dim };
if (NC_NOERR != (status = nc_def_var(fileid, "results", NC_DOUBLE,
        3, dimensions, &arrayid))) {
        ...
}
\end{verbatim}
Wenn alle Variablen eines Files definiert sind, muss der Defintion
Mode beendet werden:
\begin{verbatim}
if (NC_NOERR != (status = nc_enddef(fileid))) {
        ...
}
\end{verbatim}
Jetzt k"onnen die Daten geschrieben werden, am effizientesten mit der
Funktion \verb+nc_put_vara+. Diese erlaubt, gr"ossere Datenbl"ocke
zu schreiben.
Im Beispiel beinhalten die Arrays {\tt start} und {\tt size}
die Information, wie gross die zu schreibenden Datenbl"ocke
sein sollen:
\begin{verbatim}
size_t  start[3] = { t, 0, 0 };
size_t  size[3] = { 1, lenx, leny };
status = nc_put_vara(fileid, arrayid, start, size, data);
if (NC_NOERR != status) {
        ...
}
\end{verbatim}
\end{beispiel}

F"ur einfache Visualiserungen der Daten in einem NetCDF File kann
das Programm {\tt ncview} verwendet werden.

\subsubsection{GRIB}
\index{GRIB}
Seit numerische Wetterprognosen praktikabel sind, haben
die meteorologischen Organisationen der Welt ein Bed"urfnis,
Daten "uber den Atmosph"arenzustand austauschen zu k"onnen.
Auch diese Daten sind dreidimensional, decken jeweils einen Teil
der Erdoberfl"ache ab, und enthalten viel verschiedenen Datentypen
wie Druck, Feuchtigkeit, Bodenbedeckung, Windgeschwindigkeit, Niederschlag
und viele weitere.
"Uber die Abk"urzung GRIB besteht keine Einigkeit.
In den offiziellen Dokumenten der World Meteorological Union wird GRIB
als Abk"urzung f"ur
General Regularly-distributed Information in Binary form. 
F"ur den Rest der Welt ist GRIB eine Abk"urzung f"ur GRIdded Binary.

\subsubsection{FITS}
\index{FITS}
Oft fallen HPC-Daten in Form von Tabellen an, in diesen F"allen
kann FITS ein geeignetes Fileformat sein. FITS bedeutet ``Flexible
Image Transport System''.
Es ist vor allem in der Astronomie verbreitet, die Bilder des
Hubble-Spaceteleskop, der Europ"aischen S"udsternwarte und weiterer
Weltraumteleskope werden in diesem Format gespeichert und verbreitet.

FITS Files k"onnen Bilddaten mit jedem beliebigen Pixel-Datentyp
enthalten, im Gegensatz zum Beispiel zu JPEG oder GIF, wo Pixeldaten
immer nur 8bit sind. Wenn eine umfangreiche Simulation Bilddaten
produziert, ist es oft zweckm"assig, diese Bilder als FITS Files
abzuspeichern, um keine Bittiefe zu verlieren. Sollte sich bei
der Nachbearbeitung heraussstellen, dass Helligkeit oder Kontrast
stark angepasst werden m"ussen, ist in den FITS Files immer noch die
gesamt Bittiefe vorhanden.
Bildformate mit kleinen Pixeldatentypen machen eine Wiederholung
der Simulation notwendig.

FITS Files bestehen aus einem als Text lesbaren Header, der aus
Attribut-Wert-Paaren besteht.
Dieser Header beschreibt den Inhalt der Daten, die Anzahl der
Bildebenen, die Tabellendimensionen etc.
F"ur das Lesen und Schreiben von FITS-Daten gibt es Bibliotheken
f"ur eine Vielzahl von Programmiersprachen.
Ausserdem stehen Viewer und Browser f"ur die Arbeit mit FITS Files
zur Verf"ugung. 

FITS ist offenbar dann besonders zweckm"assig, wenn die Resultate
der Simulationen von Nutzern in Fachgebieten verwendet werden sollen,
in denen FITS ohnehin schon ein "ubliches Fileformat ist, insbesondere
in der Astronomie und Astrophysik.
Die Vatikanische Apostolische Bibliothek hat begonnen, ihren Bestand
an alten Manuskripten im FITS Format zu digitalisieren.

F"ur die Manipulation von FITS Files steht mit \texttt{cfitsio} eine effiziente
und leicht zu benutzende Bibliothek zur Verf"ugung, Source Code und
Dokumentation k"onnen von \url{http://heasarc.nasa.gov/fitsio/fitsio.html}.

\section{Visualisierung}
\rhead{Visualisierung}
Manchmal dienen Simulationen nicht der konkreten Berechnung einer
makroskopischen Gr"osse, wie dem $C_w$-Wert f"ur den Prospekt eines
Autoherstellers, sondern um Optimierungsm"oglichkeiten zu finden.
Man weiss also noch nicht, wonach man eigentlich sucht, und man
braucht m"oglichst intuitive Methoden, den ganzen Datensatz
"uberblicken zu k"onnen.
In diesem Fall ist eine geeignete Visualisierung ein erfolgversprechendes
Hilfsmittel.

Visualisierungswerkzeuge sollten in der Lage sein, aus NetCDF Files
graphische Darstellungen in zwei oder drei Dimensionen zu erzeugen.
Benutzer sollten in der Lage sein, diese Darstellungen zu manipulieren.

\subsection{ParaView}
ParaView ist ein beliebtes Open Source Visualisierungswerkzeug.
Es kann neben vielen anderen auch NetCDF Files lesen und visualisieren.
\begin{itemize}
\item Isofl"achen von Skalarfeldern.
\item Erzeugung von Stromlinien von Vektorfeldern.
\item Rechenoperationen auf verschiedenen Feldern.
\item Ausschneiden von interessanten Gebieten.
\item Scripting von erweiterten Auswertung mit Hilfe von Python, NumPy, SciPy.
\item Extraktion von Daten entlang von Geraden und Ebenen.
\item Datenfilter aus dem Visualisierungstoolkit.
\end{itemize}

Paraview kann auch auf einem grossen Cluster mit MPI parallel laufen.

Als im Februar 2013 ein Meteor in Chelyabinsk einschlug, haben unz"ahlige
Webcams, Dashcams und "Uberwachungskameras das Ereignis aufgezeichnet.
Diese Daten konnten als Input f"ur eine detaillierte Simulation
des Einschlages an den Sandia National Laboratories dienen.
Sandia verwendete daf"ur eigenen Code, der zum Beispiel auch f"ur 
die Simulation der Wirkung von Kernwaffen verwendet wird.
Die Resultate dieser Simulation wurden mit ParaView visualisiert
und zierten das Titelblatt der Zeitschrift Nature vom 14.~Novebmer 2013.

Zu den Benutzern von ParaView geh"oren auch das Supercomputer Center 
in Manno und die EPFL. 
\index{ParaView}

\subsection{Visualization Toolkit}
\index{Visualization Toolkit}
F"ur hartn"ackige F"alle mag es notwendig sein, f"ur ein spezifisches
Problem eine eigene Visualisierung zu entwickeln.
In solchen F"allen ist es unabdingbar, als Grundlage eine leistungsf"ahige
Bibliothek zur Verf"ugung zu haben. 
VTK, das Visualization Toolkit, ist die Bibliothek, mit der ParaView
realisiert wurde.
VTK wird von einigen grossen Forschungslabors als Basis f"ur die
Entwicklung ihrer eigenen Visualisierungswerkzeuge verwendet.

\subsection{Videos}
\index{Video}
Videos machen die Resultate einer komplexen Simulation auch einem
Betrachter zug"anglich, der nicht die Computer-Resourcen hat, sie sich
direkt mit einem Werkzeug wie Paraview anzuschauen. Insbeondere  sind
Videos auch gut geeignet f"ur Pr"asentationen oder f"ur Websites.

ParaView ist in der Lage, animierte Resultatpr"asentationen als eine 
Folge von Bildern zu exportieren. Bildverarbeitungswerkzeuge wie
ImageMagick k"onnen daraus MJPEG-Streams erzeugen, die mit den
\index{ImageMagick}
\index{MJPEG}
MJPEG-Tools und {\tt mpeg2enc} als MPEG Videos codiert werden k"onnen.
In dieser Form k"onnen Sie auch auf Video-Websites wie Youtube hochgeladen
werden, oder sie k"onnen in Pr"asentationen integriert werden.

Aus den Daten in einem NetCDF File kann man oft auch mit einem
einfachen Programm eine Folge von FITS Files machen, die dann von ImageMagick
und MJPEG-Tools zu einem Video verarbeit werden k"onnen.
Ein Beispiel f"ur ein solches Programm ist im Verzeichnis 
\texttt{data}
des Sourcecode-Repositories zu finden.

